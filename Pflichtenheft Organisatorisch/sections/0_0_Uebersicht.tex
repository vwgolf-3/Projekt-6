\clearpage
\section{Einleitung}\label{sec:Einleitung}
Das Organisatorische Pflichtenheft beinhaltet verschiedene Teilschritte, welche die Rahmenbedingungen definieren. In diesem werden die Projektziele, Lieferobjekte sowie die die Meilensteine festgelegt. Ausserdem beinhaltet es einen detaillierten Projektstrukturplan, welcher Arbeitspakete und den Zeitplan enthält.

\subsection{Ausgangslage}\label{subsec:Ausgangslage}

Bei einer gelungenen Gartenparty dürfen erfrischende Getränke nicht fehlen. Das Problem ist jedoch, dass kaum einer weiss, wie Cocktails gemischt werden und keiner Lust hat den ganzen Abend Barmann/Barfrau zu spielen. Hier soll nun die Cocktailmaschine für zu Hause Abhilfe schaffen und somit eine gelungene Gartenparty garantieren.

Es soll eine automatische Cocktail-Maschine entwickelt werden. Die Benutzer können wahlweise über ein Handy, einen Computer oder ein Display ihren Cocktail individuell konfigurieren. Die Cocktail-Maschine erkennt das Cocktailglas und stellt anhand der gespeicherten Serverdaten oder der auf dem Touchscreen gewählten Einstellungen das gewünschte Getränk zusammen. 
 
\newpage
\subsection{Projektziele}\label{subsec:Projektziele}

\subsubsection{Pflichtziele}\label{sec:Pflichtziele}

\begin{enumerate}

\item \underline{\textbf{Detailkonzept}}\mbox{}\\

Das Detailkonzept wird so ausgearbeitet, dass alle dazugekommenen Komponenten ebenfalls darin enthalten sind. Daraus ergibt sich folgende Liste:\\

Bestehend:
\begin{itemize}
\item Speisungen (48V, 12V, 5V, 3.3V)
\item Motor
\item ABN-Encoder
\item Endschalter
\item Motorentreiber
\item Gatetreiber
\item Durchflussmessungen
\item Pumpen
\item Display
\item Mikrocontroller\\
\end{itemize}

Dazugekommen:

\begin{itemize}
\item USB-Schnittstellen zur Programmierung des uP und des Wirelessmoduls 
\item Wirelessmodul für die Implementierung eines Web-Servers
\item RFID-Erkennung zur Bestimmung der unterschiedlichen Gläser
\item Beleuchtung der Getränkebefüllung gemäss Wunschziel\\
\end{itemize}

\item \underline{\textbf{Design der Leiterplatte}}\mbox{}\\

Die Leiterplatte soll alle Teile des Detailkonzeptes umfassen. Für das WIFI-, RFID- und Motorentreiber-Modul wird ein Development-Board verwendet. Zusätzlich zum WIFI- und RFID-Modul wird eine eigen gelayoutete Variante miteinbezogen, welche bei genügend Kapazität implementiert wird anstelle des Moduls.\\

\item \underline{\textbf{Mechanischer Aufbau der Maschine inkl. Achsensystem}}\mbox{}\\

Der mechanische Aufbau der Maschine beinhaltet folgende Teile: \\

\textbullet Rahmen \newline
\textbullet Getränkehalterung \newline
\textbullet Flüssigkeitsbeförderung \newline
\textbullet Gehäuse für Elektronik \newline
\textbullet Befestigung für Display \newline
\textbullet Glasbeförderungssystem \newline
\textbullet Überlaufwanne \newline
\textbullet Beleuchtung\\
\newpage

\item \underline{\textbf{Regler Parametrierung des Achsensystems}}\mbox{}\\

Die Regelung des Achsensystems wird mit dem TMC4671 gewährleistet. Die Regler werden so ausgelegt, dass das Glas während dem Fahren nicht überläuft. Die Bewegungsgeschwindigkeit soll jedoch auch schnell genug sein, dass der Drink in unter einer Minute hergestellt wird.\\

\item \underline{\textbf{Bediensoftware}}\mbox{}\\

Die Bediensoftware auf dem Mikrocontroller ermöglicht dem Benutzer folgende Eingaben:\\
\begin{itemize}

\item Getränkeliste mit 5 alkoholischen und 5 nicht alkoholischen Getränken, welche zur Auswahl stehen.
\item Infos zu den Getränken
\item Auswahl der Zubereitungsgrösse von 0.3l oder 0.5l
\item Nachfüllen des per Web-Servers eingestellten Getränkes mittels RFID
\item Reinigungsmodus\\
\end{itemize}

Über einen Web-Server kann der User folgende Einstellungen vornehmen: \\

\begin{itemize}
\item Auswahl des nächsten Getränkes gemäss der Getränkeliste
\item Auswahl der Zubereitungsgrösse von 0.3l oder 0.5l\\ 
\end{itemize} 


\item \underline{\textbf{Funktionstest und Analyse bezüglich der Skalierbarkeit}}\mbox{}\\
			
In einer ersten Phase wird der Print in Betrieb genommen. Dies bedeutet, dass die einzelnen Systeme mit Sonderprogrammen auf ihre Funktion geprüft werden. Dies beinhaltet die Systeme des Detailkonzeptes.\\

In einer zweiten Phase wird die Maschine auf ihre Funktion gepfüft. Dies soll die Funktionen beinhalten, welche in der Bediensoftware aufgelistet sind.\\
			
			 
\item \underline{\textbf{Software}}\mbox{}\\
\begin{itemize}
\item Die Software für den Mikrocontroller soll in C geschrieben sein. 
			
\item Für das ESP wird vorerst Arduino verwendet.\\
\end{itemize}

\item \underline{\textbf{Getränkezubereitung}}\mbox{}\\
\begin{itemize}
\item Die Abweichung der Flüssigkeitsausgabe darf höchstens 4\% betragen. \\
\end{itemize}
			
\end{enumerate}	
\newpage
\subsubsection{Wunschziele}\label{sec:Wunschziele}

\begin{enumerate}

\item \underline{\textbf{Lichtkonzept}}\mbox{}\\

Die Maschine bietet einen gewissen Showeffekt. Dazu wird ein LED-Streifen montiert, welcher die Maschine beleuchtet. Für die Beleuchtung werden RGB-LED's verwendet, was eine entsprechende Ansteuerung Hard- und Softwareseitig erfordert.\\

\item \underline{\textbf{Software}}\mbox{}\\
\begin{itemize}
\item Die Software für das ESP soll in C geschrieben sein.
\item Es soll vom Benutzer konfigurierbar sein, welches Getränk wo steht.
\item Der Benutzer soll selbst Cocktails individuell erstellen können.
\item Individuelle Anpassungen der Mischverhältnisse der gespeicherten Getränke \\ 
\end{itemize}

\item \underline{\textbf{Web-Server}}\mbox{}\\
\begin{itemize}
\item Individuelle Anpassungen der Mischverhältnisse der gespeicherten Getränke \\
\end{itemize}

\item \underline{\textbf{Regler Parametrierung des Achsensystems}}\mbox{}\\
\begin{itemize}
\item Das gewünschte Getränk soll in unter 40s zubereitet werden. \\
\end{itemize}

\item \underline{\textbf{Getränkezubereitung}}\mbox{}\\
\begin{itemize}
\item Die Abweichung der Flüssigkeitsausgabe darf höchstens 1\% betragen. \\
\end{itemize}

\end{enumerate}


\newpage

\subsection{Projektmanagement, Kommunikation, Abgabetermine, Bewertung}

Das Projekt soll von einem schlanken, ergebnisorientierten Projektmanagement begleitet werden. 
Die betreuenden Dozenten sollen periodisch (mind. alle 3 Wochen) über den Stand der Arbeiten sowie allfälliger Abweichungen zum Pflichtenheft und Projektplan informiert werden.
Es finden mindestens folgende Meetings statt: Kickoffmeeting, Besprechung Pflichtenheft/Projektvereinbarung sowie Schlusspräsentation/Verteidigung. 
Bei Bedarf können mehr Meetings durchgeführt werden.
Bezüglich Verteidigung und Bewertung gelten die Vorgaben und Richtlinien der FHNW, Hochschule für Technik.

\newpage

\subsection{Lieferobjekte}\label{subsec:Lieferobjekte}
\begin{itemize}

\item \textbf{Projektvereinbarung}\\

Per Mail,

An Projektcoach,

Bis 05.03.2020. \\

\item \textbf{Projektunterlagen (Fachbericht, Hardware, Programmcode, Schemas etc.)}\\

Per Mail, Physisch, auf USB,

An Projektcoach,

Bis 15.08.2020.\\

\item \textbf{Präsentation und Verteidigung}\\

Meeting,

In Anwesenheit von Projektcoach und Experten,

Zischen 31.08.20 und 12.09.2020.\\

\item \textbf{Fact Sheet}\\

Im LaTeX-Format inkl. Bilder und PDF (gesamter Ordner als zip-Datei),

Upload über die Projektdatenbank,

Bis spätestens 19.09.2020.\\

\item \textbf{Poster}\\

Auf Papier für Projektausstellung,

Im pptx-Format und im pdf-Format (beides in einer zip-Datei),

Upload über die Projektdatenbank,

Bis 14.08.2020.\\

\item \textbf{Räumung des Arbeitsplatzes}\\

Bis spätestens 12.09.2020


\end{itemize}	
