\clearpage
\section{Bedienung}\label{sec:Bedienung}
Der 3D-Drucker kann entweder durch das webbasierte GUI oder über ein eingebautes Display mit einem Drehgeber gesteuert werden, welche auf der Leiterplatte verbaut sind. Die jeweils verfügbaren Funktionen und Einstellungen sind identisch. Weiterhin können G-Code Dateien wahlweise mittels WLAN oder einer SD-Karte übergeben werden. 

Die Standardansicht des Displays und des GUIs bildet die Statusanzeige. Auf ihr werden aktuelle Daten wie Temperatur des Extruders und des Heizbetts, Ventilatorgeschwindigkeit, Multiplikator Geschwindigkeit, Multiplikator Zufuhr und Druckfortschritt angezeigt. Durch Druck auf den Drehgeber wird eine Liste von verschiedenen Menüs geöffnet. Diese werden als \textit{Kurzeinstellungen}, \textit{SD-Karte} und \textit{Position} bezeichnet. In \textit{Kurzeinstellungen} sind häufig verwendete Einstellungen und Funktionen zu finden, wie etwa \textit{Vorheizen ABS/PLA}, \textit{Abkühlen}, \textit{Deaktiviere Schrittmotoren}, \textit{Home All} oder \textit{Druckauftrag Abbrechen}. 

Im Menü \textit{SD-Karte} stehen die Funktionen  \textit{Mount/Unmount SD-Karte}, \textit{Drucke Datei}, \textit{Lösche Datei}. zur Verfügung. Das Menü \textit{Position} bietet verschiedene Funktionen wie etwa \textit{Bewege x,y,z}, \textit{Home x,y,z} und \textit{Home alle}, mit welchen die Achsen unabhängig voneinander bewegt und ihre Ausgangslage zurückversetzt werden können.