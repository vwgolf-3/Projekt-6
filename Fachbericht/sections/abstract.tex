\begin{abstract}
\textbf{ALT:}\\
In diesem Projekt wurde ein Konzept erstellt, um eine Cocktailmaschine zu bauen. Dies reicht von der Analyse, was für Cocktailmaschinen es bereits gibt, über die Erstellung eines Grob- und eines Detailkonzeptes bis hin zur Evaluation der Komponenten. Der Aufbau wurde so gewählt, dass ein Glas mittels eines Linearantriebes auf einem Schlitten hin- und her gefahren wird und unter dem gewünschten Flüssigkeitsauslass stehen bleibt, wo es dann befüllt wird. Die Bedienung soll über ein Touch-Display geschehen. Die Verarbeitung der Daten wird ein Mikrocontroller übernehmen. Als mechanische Komponente wird pro Zutat eine Pumpe und ein Durchflusssensor verwendet sowie ein einzelner Motor, welcher den Linearantrieb mit dem Schlitten betreibt. Als Motor wurde ein bürstenloser Gleichstrommotor verwendet, da dieser ein sehr gutes Leistungs-/Gewicht-Verhältnis aufweist und in seiner Ansteuerung sehr interessant ist. Ziel des Projekt 5 war es, anhand des Konzeptes die einzelnen Teilsysteme aufzubauen und deren Funktion zu verifizieren und zu dokumentieren. Softwaremässig wurde die Basis für den Mikrocontroller geschrieben. Dies bedeutet, dass die Teilsysteme kontrollierbar sind und im Projekt 6 ausgebaut und zusammengeführt verwendet werden können. Die Software wurde komplett in C geschrieben und ausgiebig dokumentiert. Das Resultat zeigt, dass die Komponenten zusammenpassen und der Cocktailmaschine im Projekt 6 nichts im Weg steht.


\textbf{NEU:}\\
Die in dieser Bachelorthesis entwickelte Cocktailmaschine ist ein IoT\footnote{\textbf{I}nternet \textbf{o}f \textbf{T}hings}-Produkt, welches definitionsgerecht verschiedene physikalische Elemente beinhaltet. Die Elemente sind eingebettet in Sensoren, Software und andere Technologien, welche es erlauben mit anderen Geräten zu kommunizieren. Zu den Elementen gehören ein Mikrocontroller mit zugehöriger Programmierschnittstelle, zwölf Pumpen sowie zugehörige Durchflusssensoren, ein RFID-Transponder, ein WiFi/Bluetooth-Modul mit zugehöriger Programmierschnittstelle, eine SD-Karte, ein FOC-Motorentreiber mit zugehörigem Kraft-Teil, ein Display und eine RGBW-LED-Regelung. Anhand der ersten Erfahrungen im Projekt fünf wurde eine Leiterplatine gelayoutet, welche die erwähnten Elemente verbindet. Über eine Android-Applikation wird den Benutzern die Möglichkeit geboten, einen neuen Cocktail zu erstellen oder das Lieblingsgetränk auszuwählen. Das gewählte Getränk kann vor Ort über den RFID-Tag bestellt werden. Die Mechanik besteht aus einem Aluminiumgerüst, welches verschalt wurde. Zusätzlich wurde eine Styroporbox um die Zutaten gebaut. Ein durch einen Motor angetriebener Schlitten fährt das Glas unter den gewünschten Flüssigkeitsauslass, wo das Glas befüllt wird. Die Dokumentation enthält eine Beschreibung der Platine und der darauf befindenden Elementen, eine Anleitung zur Inbetriebnahme der Elemente und einen Software-Teil. Das Produkt wurde physikalisch abgegeben und ist voll funktionsfähig.

\end{abstract}