\begin{abstract}

In diesem Projekt wurde ein Konzept erstellt, um eine Cocktailmaschine zu bauen. Dies reicht von der Analyse, was für Cocktailmaschinen es bereits gibt, über die Erstellung eines Grob- und eines Detailkonzeptes bis hin zur Evaluation der Komponenten. Der Aufbau wurde so gewählt, dass ein Glas mittels eines Linearantriebes auf einem Schlitten hin- und her gefahren wird und unter dem gewünschten Flüssigkeitsauslass stehen bleibt, wo es dann befüllt wird. Die Bedienung soll über ein Touch-Display geschehen. Die Verarbeitung der Daten wird ein Mikrocontroller übernehmen. Als mechanische Komponente wird pro Zutat eine Pumpe und ein Durchflusssensor verwendet sowie ein einzelner Motor, welcher den Linearantrieb mit dem Schlitten betreibt. Als Motor wurde ein bürstenloser Gleichstrommotor verwendet, da dieser ein sehr gutes Leistungs-/Gewicht-Verhältnis aufweist und in seiner Ansteuerung sehr interessant ist. Ziel des Projekt 5 war es, anhand des Konzeptes die einzelnen Teilsysteme aufzubauen und deren Funktion zu verifizieren und zu dokumentieren. Softwaremässig wurde die Basis für den Mikrocontroller geschrieben. Dies bedeutet, dass die Teilsysteme kontrollierbar sind und im Projekt 6 ausgebaut und zusammengeführt verwendet werden können. Die Software wurde komplett in C geschrieben und ausgiebig dokumentiert. Das Resultat zeigt, dass die Komponenten zusammenpassen und der Cocktailmaschine im Projekt 6 nichts im Weg steht.

\end{abstract}