\begin{abstract}

Die in dieser Bachelorthesis entwickelte Cocktailmaschine ist ein IoT\footnote{\textbf{I}nternet \textbf{o}f \textbf{T}hings}-Produkt, welches definitionsgerecht verschiedene physikalische Elemente beinhaltet. Die Elemente sind eingebettet in Sensoren, Software und andere Technologien, welche es erlauben mit anderen Geräten zu kommunizieren. Zu den Elementen gehören ein Mikrocontroller mit zugehöriger Programmierschnittstelle, zwölf Pumpen sowie zugehörige Durchflusssensoren, ein RFID-Transponder, ein WiFi/Bluetooth-Modul mit zugehöriger Programmierschnittstelle, eine SD-Karte, ein FOC-Motorentreiber mit zugehörigem Kraft-Teil, ein Display und eine RGBW-LED-Regelung. Anhand der ersten Erfahrungen im Projekt 5 wurde eine Leiterplatine gelayoutet, welche die erwähnten Elemente verbindet. Über eine Android-Applikation wird den Benutzern die Möglichkeit geboten, einen neuen Cocktail zu erstellen oder das Lieblingsgetränk auszuwählen. Das gewählte Getränk kann dann vor Ort über den RFID-Tag bestellt werden. Die Mechanik besteht aus einem Aluminiumgerüst, welches verschalt wurde. Zusätzlich wurde zur Kühlhaltung eine Styroporbox um die Zutaten gebaut. Ein durch einen Motor angetriebener Schlitten fährt das Glas unter den gewünschten Flüssigkeitsauslass, wo das Glas befüllt wird. Die Dokumentation enthält eine Beschreibung der Platine und der darauf befindenden Elementen, eine Anleitung zur Inbetriebnahme der Elemente und einen Software-Teil. Das Produkt wurde physikalisch abgegeben und ist voll funktionsfähig.

\todo{Voll funktionsähig...}

\end{abstract}