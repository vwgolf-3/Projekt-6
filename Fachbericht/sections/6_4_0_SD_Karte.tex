\subsubsection{mikroSD-Karte}
\label{subsubsec:Inbetriebnahme_mikroSD_Karte}

Vorgehen:
\begin{enumerate}
\item Benötigte Applikation aus dem Software-Ordner auf dem USB-Stick in Atmel Studio öffnen.\\
\textcolor{magenta}{Software\textrightarrow Atmega\textrightarrow  2\_Testapplikation\_SD\_Karte \textrightarrow SD\_Karte\_Testapplikation}\\

\item Software hochladen:\\
\textcolor{blue}{AtmelStudio \textrightarrow Tools \textrightarrow Cocktailmixer}\\

\item Programm für Kommunikation über die serielle Schnittstelle downloaden (z.B HTerm 0.8.1beta.)\cite{hammer_hterm_nodate}\\
\item Verbindung mit Mikrocontroller herstellen und Neustart über Button DTR auslösen.\\

\begin{table}[h!]
\center
\begin{tabular}{lcl}
Port & = & \textcolor{red}{COMx} \\
Baudrate & = & 9600 \\
Data & = & 8 \\
Stop & = & 1 \\
\end{tabular}
\end{table}

Der Port \textcolor{red}{COMx} ist aus dem Geräte-Manager zu entnehmen. Es ist derselbe Port, welcher verwendet wird um die Software hochzuladen.\\

\item Ob eine SD-Karte gefunden wurde, ist erkennbar, wenn die Version der SD-Karte angezeigt wird. Jetzt sollte sich die SD-Karte lesen und beschreiben lassen.

\end{enumerate}
