\subsubsection{mikroSD-Karte}
\label{subsubsec:Inbetriebnahme_mikroSD_Karte}

Vorgehen:
\begin{enumerate}
\item Benötigte Applikation, welche im Software-Ordner auf dem USB-Stick oder Github \cite{aebi_projekt-6softwareatmega_2020} zu finden ist, in Atmel Studio öffnen.\\
\textcolor{magenta}{Software\textrightarrow Atmega\textrightarrow  2\_SD\_Karte \textrightarrow SD\_Karte}\\

\item Software hochladen:\\
\textcolor{blue}{AtmelStudio \textrightarrow Tools \textrightarrow PartyMixer}\\

\item Programm für Kommunikation über die serielle Schnittstelle downloaden (z.B HTerm 0.8.1beta.)\cite{hammer_hterm_nodate}\\
\item Verbindung mit Mikrocontroller herstellen und Neustart über Button DTR auslösen. Damit wird der DTR-Pin getoggelt und der Mikrocontroller neu gestartet\\

\begin{table}[h!]
\center
\begin{tabular}{lcl}
Port & = & \textcolor{red}{COMx} \\
Baudrate & = & 57600 \\
Data & = & 8 \\
Stop & = & 1 \\
\end{tabular}
\end{table}

Der Port \textcolor{red}{COMx} ist aus dem Geräte-Manager zu entnehmen. Es ist derselbe Port, welcher verwendet wird um die Software hochzuladen.\\

\item Wird eine SD-Karte gefunde, wird im seriellen Display ''Boot...OK!'' angezeigt. Mit der Testapplikation wird bei erfolgreichem Booten ein File namens \textit{TEST.txt} mit dem Textinhalt \textit{Hallo File} erstellt.

\end{enumerate}
