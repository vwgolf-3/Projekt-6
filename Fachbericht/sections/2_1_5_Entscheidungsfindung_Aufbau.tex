\subsection{Entscheidungfindung des Aufbaus}\label{subsec:Entscheidungfindung_des_Aufbaus}

Für die Entscheidungsfindung des Aufbaus wird eine Tabelle \ref{tab:Cocktailmaschinen_Vergleich} erstellt, welche die Vor- und Nachteile der Maschinen aufzeigt.
\newline

\begin{table}[h!]
\begin{tabularx}{\textwidth}{|l|X|X|}
\hline

\textbf{Cocktailmaschine:} & \textbf{Vorteile:} & \textbf{Nachteile:} 
\\
\hline

Spirits &

\begin{itemize}[leftmargin=0.3cm,label={--}]
\item Schnelle Zubereitung (ca. 20s)
\item Mobiler Aufbau
\item Modularer Aufbau
\item Einfache Bedienung via App
\item Abspielen von Musik
\item Einfache Reinigung
\end{itemize}   &

\begin{itemize}[leftmargin=0.3cm,label={--}]
\item Kein Showeffekt für Befüllung
\end{itemize}   \\
\hline

Der Cocktailmixer & 

\begin{itemize}[leftmargin=0.3cm,label={--}]
\item Sehr schnelle Zubereitung (ca. 3-5s)
\item Mobiler Aufbau auf Rollen
\item Einfache Bedienung via Touchpad
\item Riesige Getränkeauswahl
\item Grosses Fassungsvolumen
\end{itemize}   & 

\begin{itemize}[leftmargin=0.3cm,label={--}]
\item Kein Showeffekt für Befüllung
\item Grosse, sperrige Maschine
\item Keine Appsteuerung
\item Grosser Reinigungsaufwand
\item Kein Showeffekt
\end{itemize}   \\
\hline

Cocktail Avenue   & 

\begin{itemize}[leftmargin=0.3cm,label={--}]
\item Schnelle Zubereitung (ca. 25s)
\item Einfache Bedienung via Touchpad
\item Einfacher Flaschenwechsel
\item Präzise Dosierung
\item Grosser Showeffekt durch Schlitten
\item Einfache Reinigung
\end{itemize}   &

\begin{itemize}[leftmargin=0.3cm,label={--}]
\item Längere Zubereitungszeit durch Bewegung
\item Benötigt viel Platz
\item Stark begrenzte Getränkeauswahl
\end{itemize}   \\
\hline

myRocktail &

\begin{itemize}[leftmargin=0.3cm,label={--}]
\item Grosser Showeffekt durch Roboterarm
\item Einfache Bedienung via Touchpad
\item Einfacher Flaschenwechsel
\item Präzise Dosierung
\item Mobiler Aufbau auf Rollen
\item Einfache Reinigung
\end{itemize}   &

\begin{itemize}[leftmargin=0.3cm,label={--}]
\item Längere Zubereitungszeit durch Bewegung (ca. 80s)
\item Benötigt viel Platz
\item Stark begrenzte Getränkeauswahl
\end{itemize}   \\
\hline
\end{tabularx}
\caption{Cocktailmaschinen Vergleich}
\label{tab:Cocktailmaschinen_Vergleich}
\end{table}
\newpage
Die Cocktailmaschine soll einige Anforderungen erfüllen. Aus diesen resultiert dann die komplette Entscheidungsfindung bezüglich des Aufbaus. Dies ist zum einen der Showeffekt. Die Cocktailmaschine soll ein Hingucker sein und dem Benutzer etwas für das Auge bieten. Daher soll das Glas zu den verschiedenen Befüllgetränken befördert werden, um diese zu befüllen. Ausserdem soll es weniger als eine Minute dauern, bis ein Cocktail erstellt ist. Da der Roboterarm weit mehr als eine Minute benötigt um einen Cocktailzu erstellen, führt zur Entscheidung ein Förderband einzusetzen, wie es bei der Cocktail Avenue der Fall ist. Weiter soll dem Benutzer eine möglichst Benutzerfreundliche Bedienoberfläche geboten werden. Diese soll so intuitiv wie möglich gestaltet werden. Daher wird ein Touchscreen verbaut, wie es beim Cocktailmixer und der myRocktail der Fall ist. Von der Spirits soll die Flüssigkeitsbeförderung übernommen werden. Daher werden Pumpen, sowie Durchflussmessgeräte verbaut.   