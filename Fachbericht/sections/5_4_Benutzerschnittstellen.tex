\subsection{Benutzerschnittstellen}
\label{subsec:Benutzerschnittstellen}

Über die Benutzerschnittstellen geschehen die Interaktionen zwischen Mensch und Maschine. Im Gesamten gibt es drei verschiedene Schnittstellen, das Display, das WiFi-Modul und das RDIF-Modul. Die Hauptinteraktion geschieht über das Display, über das WiFi-Modul lassen sich mit der Android-Applikation einige Funktionen der Maschine von extern aufgerufen werden, mit dem RFID-Modul kann ein Zugewiesener Cocktail direkt ausgewählt werden. Die zugehörigen Schaltungen werden im Folgenden beschrieben.

%Die vorkommenden Interfaces sind mit ihren Funktionen tabellarisch aufgelistet:
%
%\begin{tabularx}{\textwidth}{|l|X|}
%\hline
%\textbf{Interface} & \textbf{Funktion}\\
%\hline
%Display & Direkte Steuerung zur Cocktailherstellung und Konfiguration der Maschine mittels GUI auf dem Display. \\
%\hline
%ESP32 & Indirekte Steuerung zur Cocktailerstellung und RFID Tag Zuweisung der Maschine mittels GUI im PartyMixer App.\\
%\hline
%RFID & Abrufen des per Android App oder Display hinterlegten Getränkes mittels RFID Tag. Mengenauswahl per Display.\\
%\hline
%\end{tabularx}
%
% 