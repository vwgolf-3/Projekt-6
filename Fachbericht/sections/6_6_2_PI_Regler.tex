\subsubsection{PI-Regler}
\label{subsubsec:PI_Regler}

Die Regler, welche im FOC-Treiber integriert sind, werden verwendet, um einen Motor an eine bestimmte Position zu fahren, mit einer bestimmte Geschwindigkeit drehen oder ein bestimmtes Drehmoment anzulegen. Im Gegensatz zu einem Motor, dem nur eine Spannung vorgegeben wird, kann bei einer Abweichung des vorgegebenen Parameters nachgeregelt werden. So kann beispielsweise bei grösserer Belastung die angelegte Spannung erhöht werden, sodass die vorgegebene Geschwindigkeit gehalten werden kann.

Für die Regelung eines Motors können verschieden aufgebaut sein. Die Regelstruktur, wie sie im FOC-Treiber TMC4671 integriert ist, ist eine Kaskadenregelung. Eine Kaskadenregelung besteht in der Regel aus drei überlagerten Regelkreisen. Der innerste Regelkreis ist der Stromregelkreis. Dieser ist dem Geschwindigkeitsregelkreis unterlagert. Der Geschwindigkeitsregelkreis ist wiederum den Positionsregelkreis unterlagert. Bei einer Kaskadenregelung ist die Ausgangsgrösse des überlagerten Regelkreises die Eingangsgrösse des unterlagerten Regelkreises. Aus stabilitätsgründen ist darauf zu achten, dass die Nachstellzeit der äusseren Regelkreise grösser ist, alsi die der Inneren.

Die Regelung im TMC4671 besteht aus einem innersten, schnellen Stromregelkreis. Dieser soll in der Lage sein, schnellst möglich mit einer Erhöhung des Drehmomentes auf eine Abweichung zu reagieren. Ausserdem gibt der Stromregelkreis die Kommutierung des Motors vor.

Der Geschwindigkeitsregelkreis ist dem Stromregelkreis überlagert. Weicht die aktuelle Geschwindigkeit von der vorgegebenen Geschwindigkeit ab, so gibt der Geschwindigkeitsregelkreis dem Stromregelkreis eine Abweichung vor, wodurch der Strom durch die Spulen und somit das Drehmoment erhöht oder gesenkt wird.

Der Positionsregelkreis ist dem Geschwindigkeitskreis überlagert. Weicht die aktuelle Position von der vorgegebenen Position ab, so gibt der Positionsregelkreis dem Geschwindigkeitsregelkreis eine Abweichung vor, wodurch der Geschwindigkeitsregelkreis in die erforderliche Richtung korrigiert.

Mit der Begrenzung der Eingangsgrössen wird die Mechanik geschont und der Motor vor Überlast geschützt. Dies wird erreicht, indem die Ausgangsgrössen der PI-Regler mit einem Limit versehen werden. So kann der Strom durch die Spulen begrenzt werden, eine Maximalgeschwindigkeit definiert werden und die Enden der Positionen vorgegeben werden. Die Limits sind auf den Motor abgestimmt. \cite{stahl_simulation_2014}

Die Werte der PI-Regler wurden Experimentell bestimmt. Folgende Werte wurden dabei ermittelt:

\begin{tabularx}{\linewidth}{|l|X|X|l|}
\hline
\textbf{Regelkreis} & \textbf{P-Anteil} & \textbf{I-Anteil} & \textbf{Nachstellzeit}\\
\hline
Drehmoment & 100 & 1200 & \\
\hline
Fluss & 100 & 1200 & \\
\hline
Geschwindigkeit & 2000 & 300 & \\
\hline
Position & 100 & 0 & \\
\hline
\end{tabularx}

Die Limits wurden folgendermassen gesetzt:

\begin{tabularx}{\linewidth}{|l|X||l|X|}
\hline
\textbf{Limit} & \textbf{Wert} & \textbf{Limit} & \textbf{Wert}\\
\hline
Drehmoment & 2500 & Fluss & 2500\\
\hline
Geschwindigkeit & 1500 &  & \\
\hline
\end{tabularx}

Die Regler wurden so bestimmt, dass die Regelkreise in unbelasteten Zustand ihre Sollwerte schnellst möglich erreichen, ohne überzuschiessen. Dies Betrifft den Regler für den Stromregelkreis und den Regler für den Geschwindigkeitsregelkreis. Die zugehörigen Parameter wurden in der TMCL-IDE ermittelt. 

Mit diesen Werte ist es nur schwierig realisierbar, den Schlitten mit einer kontrollierten Bewegung fortzubewegen. Um dennoch gezielt Positionen anzufahren, wurde eine Software-Ramp geschrieben. Diese berechnet den gewünschten Weg unter Berücksichtigung der maximalen Beschleunigung und der maximalen Geschwindigkeit. Durch eine periodische Iteration in Millisekunden-Schritten über die Zeit gibt die Ramp dem FOC-Treiber schrittweise die berechnete Position vor. Die Endposition, die Geschwindigkeit und die Beschleunigung der Ramp sind einfach skallierbar, was eine schnellere Einbettung in das Gesamtsystem ermöglicht.