\subsubsection{Software-Ramp}
\label{subsubsec:Software_Ramp}

Mit den in Kapitel \ref{subsubsec:Inbetriebnahme_ABN-Encoder} ermittelten Werte ist es nur schwierig realisierbar, den Schlitten mit einer kontrollierten Bewegung fortzubewegen. Um dennoch gezielt Positionen anzufahren, wurde eine Software-Ramp geschrieben. Diese berechnet den gewünschten Weg unter Berücksichtigung der maximalen Beschleunigung und der maximalen Geschwindigkeit. Durch eine periodische Iteration in Millisekunden-Schritten über die Zeit gibt die Ramp dem FOC-Treiber schrittweise die berechnete Position vor. Die Endposition, die Geschwindigkeit und die Beschleunigung der Ramp sind einfach skalierbar, was eine schnellere Einbettung in das Gesamtsystem ermöglicht. Ein Matlab-Script war Inspiration für die Software Ramp. Da dieses Script Funktionen enthält, welche nur schwer mit dem Mikrocontroller umgesetzt werden können. Deshalb wurde zur Vorbereitung ein eigenes Matlab-Script geschrieben, mit welchem der Algorithmus in C übersetzt werden kann. Dieses ist im Anhang Kapitel \ref{Appendix:Ramp_Matlab_Script} zu sehen. Es ist zu beachten, dass der Jerk nicht in die Software des Mikrocontrollers implementiert wurde. Einen theoretischen Teil zur Software Ramp ist im Anhang \ref{} zu finden. \cite{bearee_gentrajm_2007}

Vorgehen:
\begin{enumerate}
\item Benötigte Applikation, welche im Software-Ordner auf dem USB-Stick oder Github \cite{aebi_projekt-6softwareatmega_2020} zu finden ist, in Atmel Studio öffnen.\\
\textcolor{magenta}{Software\textrightarrow Atmega\textrightarrow 5\underline{ }Motor\underline{ }Linear\underline{ }Ramp\textrightarrow 1\underline{ }Motor\underline{ }Testsoftware\textrightarrow Motor}\\

\item Software hochladen:\\
\textcolor{blue}{AtmelStudio\textrightarrow Tools\textrightarrow PartyMixer}\\

\item Über die serielle Schnittstelle kann nun der Motor mit diversen Eingaben bewegt werden. Im File sind weitere Informationen zu finden.

\end{enumerate}