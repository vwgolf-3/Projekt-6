\subsubsection{Software-Ramp}
\label{subsubsec:Software_Ramp}

Mit den in Kapitel \ref{subsubsec:Inbetriebnahme_ABN-Encoder} ermittelten Werte ist es nur schwierig realisierbar, den Schlitten mit einer kontrollierten Bewegung fortzubewegen. Um dennoch gezielt Positionen anzufahren, wurde eine Software-Ramp geschrieben. Diese berechnet den gewünschten Weg unter Berücksichtigung der maximalen Beschleunigung und der maximalen Geschwindigkeit. Durch eine periodische Iteration in Millisekunden-Schritten über die Zeit gibt die Ramp dem FOC-Treiber schrittweise die berechnete Position vor. Die Endposition, die Geschwindigkeit und die Beschleunigung der Ramp sind einfach skalierbar, was eine schnellere Einbettung in das Gesamtsystem ermöglicht.


Vorgehen:
\begin{enumerate}
\item Benötigte Applikation aus dem Software-Ordner auf dem USB-Stick in Atmel Studio öffnen.\\
\textcolor{magenta}{Software\textrightarrow Atmega\textrightarrow 5\underline{ }Motor\underline{ }Linear\underline{ }Ramp\textrightarrow 1\underline{ }Motor\underline{ }Testsoftware\textrightarrow Motor}\\

\item Software hochladen:\\
\textcolor{blue}{AtmelStudio\textrightarrow Tools\textrightarrow PartyMixer}\\

\item Über die serielle Schnittstelle können nun 

\end{enumerate}
\todo{Cite: GenTraj}
\todo{Fertig schreiben Ramp implementierung}