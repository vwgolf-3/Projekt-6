\subsubsection{Software-Ramp}
\label{subsubsec:Software_Ramp}
 Da der ABN-Encoder am FOC-Treiber angeschlossen ist, wird die Schnittstelle für den ABN-Encoder am FOC-Treiber initialisiert. Dazu müssen weitere Register gesetzt werden, worunter auch die Register für die PI-Regler fallen. Die Software TMCL-IDE hilft bei der Bestimmung der Parameter für den ABN-Encoder und PI-Regler. Welche Register wie beschrieben wurden ist im Anhang Kapitel \ref{Appendix:ABN_Register} zu sehen. Die Initialisierung sowie das Auslesen gewisser Register ist mit der Testapplikation ''\textit{4\underline{ }Motor\underline{ }ABN\underline{ }Encoder}'' möglich.

Die Software-Ramp wird mit einem Testprogramm in Betrieb genommen. Das Programm initialisiert die benötigte Hardware des Mikrocontrollers (UART, SPI, Pins), enthält die Library des FOC-Treibers, des Gate-Treibers und der Software-Ramp und ermöglicht das Debugen über die serielle Schnittstelle. Durch eine periodische Iteration über die Zeit gibt die Ramp dem FOC-Treiber schrittweise die berechnete Position vor. Die Endposition, die Geschwindigkeit und die Beschleunigung der Ramp sind einfach skalierbar, was eine schnellere Einbettung in das Gesamtsystem ermöglicht. Das Matlab-Script \textit{GenTraj.m} von Richard Bearee war Grundlage für die Software Ramp. Das Script enthält Funktionen, welche nur schwer mit dem Mikrocontroller umzusetzen sind. Deshalb wurde zur Vorbereitung ein eigenes Matlab-Script geschrieben, mit welchem der Algorithmus in C übersetzt werden kann. Dieses ist im Anhang Kapitel \ref{Appendix:Ramp_Matlab_Script} zu sehen. Es ist zu beachten, dass der Jerk nicht in die Software des Mikrocontrollers implementiert wurde. Einen theoretischen Teil zur Software Ramp ist im Anhang Kapitel \ref{Appendix:Software_Ramp_Theorie} zu finden.

Es wird erwartet, dass sich der Motor mit der Software-Ramp langsam und kontrolliert dreht. Dazu muss über den seriellen Monitor bei leerem Textfeld Enter eingegeben werden.

Vorgehen:
\begin{enumerate}
\item Benötigte Applikation, welche im Software-Ordner auf dem USB-Stick oder Github \cite{aebi_projekt-6softwareatmega_2020} zu finden ist, in Atmel Studio öffnen.\\
\textcolor{magenta}{Software\textrightarrow Atmega\textrightarrow 5\underline{ }Motor\underline{ }Linear\underline{ }Ramp\textrightarrow 1\underline{ }Motor\underline{ }Testsoftware\textrightarrow Motor}\\

\item Software hochladen:\\
\textcolor{blue}{AtmelStudio\textrightarrow Tools\textrightarrow PartyMixer}\\

\item Über die serielle Schnittstelle kann nun der Motor mit diversen Eingaben bewegt werden. Im File sind weitere Informationen zu finden.

\end{enumerate}

Ergebnis: Nach Übermitteln der Symbole aus dem seriellen Port an den Mikrocontroller beginnt der Motor langsam und kontrolliert mit kontinuierlicher Beschleunigung zu drehen, bis die Maximalgeschwindigkeit erreicht wurde. Vor Erreichen der Endposition verlangsamt der Motor mit kontinuierlicher negativer Beschleunigung.