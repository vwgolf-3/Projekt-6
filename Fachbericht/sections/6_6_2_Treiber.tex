\subsubsection{FOC-Treiber}
\label{subsubsec:Inbetriebnahme_FOC_Treiber}

Der FOC-Treiber wird über die SPI-Schnittstelle in Betrieb genommen. Dazu werden die Parameter verwendet, welche aus der TMCL-IDE verwendet werden. Die Standardparameter beinhalten Informationen zum Motor, zwei Sekunden Linksdrehung im Open-Loop, 2 Sekunden Rechtsdrehung im Open-Loop und dann Stop. Welche Register wie beschrieben werden, ist im Anhang Kapitel \ref{Appendix:TMC4671_Register} zu ersichtlich. Die Initialisierung sowie das Auslesen gewisser Register ist mit der Testapplikation ''\textit{3\underline{ }Motor\underline{ }Openloop}'' möglich, welche im Softwareordner auf dem USB-Stick zu finden ist.
\todo{Abgabe USB-Stick/Github-Account ??}
Das Setup, mit welchem der Treiber softwareseitig in Betrieb genommen wurde, ist im Anhang Kapitel \ref{Appendix:TMC4671_Setup} gezeigt.

Vorgehen:
\begin{enumerate}
\item Benötigte Applikation aus dem Software-Ordner auf dem USB-Stick in Atmel Studio öffnen.\\
\textcolor{magenta}{Software\textrightarrow Atmega\textrightarrow 3\underline{ }Motor\underline{ }Openloop\textrightarrow 1\underline{ }Motor\underline{ }Testsoftware\textrightarrow Motor}\\

\item Software anpassen: Zeile 23 bis 32\\
\textcolor{OliveGreen}{
	initTMC4671\underline{ }Openloop();\\
\\
    while (1) \\
    \{\\
		\underline{ }delay\underline{ }ms(10000);\\
		read\underline{ }registers\underline{ }TMC4671();\\
    \}
}\\

\item Software hochladen:\\
\textcolor{blue}{AtmelStudio\textrightarrow Tools\textrightarrow PartyMixer}\\

\item SPI-Kommunikation und ausgehende Gate-Signale vom FOC-Treiber zum Gate-Treiebr mit Oszilloskop überprüfen. Im Anhang Kapitel \ref{Appendix:TMC4671_SPI} sind die Messbilder zur SPI- Kommunikation zu finden und in Anhang \ref{Appendix:TMC4671_Gate_Ctrl} die Messbilder zur Gate-Control vom TMC4671 zum TMC6200.
\end{enumerate}

\todo{Skripts nicht benutzen in der Maschine.}