\newpage
\subsubsection{ABN-Encoder}
\label{subsubsec:Inbetriebnahme_ABN-Encoder}

Der ABN-Encoder wird an den FOC-Treiber angeschlossen. Daher wird die Schnittstelle am TMC4671 initialisiert. Dazu werden die Parameter verwendet, welche in der TMCL-IDE verwendet wurden. Das Einlesen wird getestet, indem der Treiber in den Positionsmodus versetzt wird, wozu das Feedback des Rotors nötig ist. Hat die Initialisierung funktioniert, dreht sich der Motor an eine gewünschte Position und haltet dort. Zur Initialisierung gehören auch einige Einstellungen zu den PI-Reglern. Die Initialisierung sowie das Auslesen gewisser Register ist mit der Tesapplikation ''\textit{4\underline{ }Motor\underline{ }ABN\underline{ }Encoder}'' möglich.

Das bisherige Setup hat sich nun um den ABN-Encoder erweitert und ist im Anhang Kapitel \ref{Appendix:ABN_Setup} ersichtlich.

Vorgehen:
\begin{enumerate}
\item Benötigte Applikation aus dem Software-Ordner auf dem USB-Stick in Atmel Studio öffnen.\\
\textcolor{magenta}{Software\textrightarrow Atmega\textrightarrow 4\underline{ }Motor\underline{ }ABN\underline{ }Encoder\textrightarrow 1\underline{ }Motor\underline{ }Testsoftware\textrightarrow Motor}\\

\item Software hochladen:\\
\textcolor{blue}{AtmelStudio\textrightarrow Tools\textrightarrow Cocktailmixer}\\

\item Ausgehende Signale des ABN-Encoders überprüfen. Im Anhang Kapitel \ref{Appendix:ABN_Signale} ist das Messbild des ABN-Encoder-Ausgangs zu sehen.

\end{enumerate}