\subsection{Beleuchtung}
\label{subsec:Inbetriebnahme_Beleuchtung}

Die Farbe und Helligkeit der Beleuchtung wird mit einem PWM-Signal geregelt. Das PWM-Signal wird erzeugt mittels Timer Interrupts, damit Regelung parallel neben dem Programfluss geschehen kann. Insgesamt benötigt es fünf Timer, um das LED-Band anzusteuern. Jeweils einen für jede Farbe und einen um durch den Rainbow-Modus zu interieren.

Die Inbetriebnahme wurde anhand des Datenblatts und des Tutorials von Atmel Mikrochip gemacht.\cite{mikrochip_makes_getting_2015}\cite{mikrochip_makes_getting_2015-1}\cite{mikrochip_makes_getting_2015-2}. Im Anhang Kapitel \ref{Appendix:Timer_PWM} ist der detailierte Vorgang zur Initialisierung des PWM-Signals beschrieben. Aus der Beschreibung geht hervor, dass das Signal mit einer Frequenz von 10kHz getaktet wird und der Duty-Cycle nicht über 95\% gehen sollte. Ausserdem wird im erwähnten Kapitel der Rainbow-Algorithmus beschrieben.



\subsubsection{Vorgehen}

\begin{enumerate}
\item Benötigte Applikation aus dem Software-Ordner auf dem USB-Stick in Atmel Studio öffnen.\\
\textcolor{magenta}{Software\textrightarrow Atmega\textrightarrow 6\underline{ }LED\underline{ }Control\textrightarrow 1\underline{ }LED\underline{ }Testsoftware\textrightarrow LED\underline{ }Control}\\

\item Software hochladen:\\
\textcolor{blue}{AtmelStudio\textrightarrow Tools\textrightarrow Cocktailmixer}\\

\end{enumerate}