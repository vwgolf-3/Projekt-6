\subsection{Beleuchtung}
\label{subsec:Inbetriebnahme_Beleuchtung}

Die Inbetriebnahme der Beleuchtung geschieht mit einem Testprogramm. Das Programm initialisiert die benötigte Hardware des Mikrocontrollers (UART, Pins, Interrupts), enthält eine LED-Library und ermöglicht das Debugen über die serielle Schnittstelle.

Die Farbe und Helligkeit der Beleuchtung wird mit einem PWM-Signal geregelt. Das PWM-Signal wird erzeugt mittels Timer Interrupts, damit Regelung parallel neben dem Programfluss geschehen kann. Insgesamt benötigt es fünf Timer um das LED-Band anzusteuern. Jeweils einen für jede Farbe und einen um durch den Rainbow-Modus zu interieren.

Die Inbetriebnahme wurde anhand des Datenblatts und des Tutorials von Atmel Mikrochip gemacht, welches auf Youtube zu finden ist.\cite{mikrochip_makes_getting_2015}\cite{mikrochip_makes_getting_2015-1}\cite{mikrochip_makes_getting_2015-2}. Im Anhang Kapitel \ref{Appendix:Timer_PWM} sind Auszüge aus dem Datenblatt und eine Beschreibung des Auszugs zu finden. Das Kapitel zeigt, dass das PWM-Signal mittels Timer-Interrupts erzeugt wird. Um das Signal mit 10kHzr und variablem Duty-Cycle zu Takten, wird der Timer im CTC-Mode betrieben.

Es wird erwartet, dass das LED-Band zu Beginn Rot leuchtet und dann durch alle Farben wechselt. Wird über den seriellen Port eine '1' geschrieben, so muss das Licht vom Rainbow-Modus zu Weiss wechseln. Wird eine '0' geschrieben, so muss das Licht von Weiss zum Rainbow-Modus wechseln.

Vorgehen:

\begin{enumerate}
\item Benötigte Applikation, welche im Software-Ordner auf dem USB-Stick oder Github \cite{aebi_projekt-6softwareatmega_2020} zu finden ist, in Atmel Studio öffnen.\\
\textcolor{magenta}{Software\textrightarrow Atmega\textrightarrow 6\underline{ }LED\underline{ }Control\textrightarrow 1\underline{ }LED\underline{ }Testsoftware\textrightarrow LED\underline{ }Control}\\

\item Software hochladen:\\
\textcolor{blue}{AtmelStudio\textrightarrow Tools\textrightarrow PartyMixer}\\

\end{enumerate}

Ergebnis: Die LEDs leuchten zu Beginn Rot und wechseln dann durch die Farben. Wird eine '1' eingegeben, wechselt das Licht von Rainbow zu Weiss, wird eine '0' eingegeben, wechselt das Licht von Weiss nach Rainbow.
Die Inbetriebnahme hat gezeigt, dass der Duty-Cycle nicht über 95\% gehen sollte, da die LEDs ansonsten flimmern.