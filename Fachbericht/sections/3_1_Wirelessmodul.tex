\subsection{Wirelessmodul}
\label{subsec:Wirelessmodul}

Über einen Web-Host soll der User die Möglickeit haben, Getränke auszuwähen und seinem RFID-Chip zuzuordnen, sowie diverse kleinere Einstellungen an der Maschine vorzunemen. Dazu wird ein WiFi-Modul benötigt.

Aufgrund schon bestehender Erfahrungen wurde ein Espressif ESP-Modul ausgewählt. Grundsätzlich standen zwei Modelle zur Auswahl. Das ESP8266 und das ESP32. Für die Cocktailmaschine wurde das ESP32 ausgewählt, da dies einfach Leistungsstärker ist. Die genauen Datenvergleiche sind in Tabelle \ref{tab:ESP} ersichtlich.

\begin{table}[!h]
\begin{tabularx}{\textwidth}{l|X|X}
MCU                    & Xtensa Single-core 32-bit L106 & Xtensa Dual-Core 32-bit LX6 \\ \hline
802.11 b/g/n     		& HT20                           & HT40                                       \\
Bluetooth              	& No                             & Bluetooth 4.2 and BLE                      \\
Arbeitsfrequenz			& 80 MHz                         & 160 MHz                                    \\
SRAM                   	& No                             & Yes                                         \\
Flash                  	& No                             & Yes                                          \\
GPIO                   	& 17                             & 36                                         \\
SPI/I2C/I2S/UART       	& 2/1/2/2                        & 4/2/2/2                                    \\
ADC                    	& 10-bit                         & 12-bit                                     \\
Ethernet Interface 		& No                             & Yes                                          \\
Touchsensor           	& No                             & Yes                                          \\
Temperatursensor     	& No                             & Yes                                          \\
Hall-Sensor     		& No                             & Yes \\
Arbeitstemperatur    	& -40ºC to 125ºC                 & -40ºC to 125ºC                             \\
Price                  	& \$ (3\$ - \$6)                   & \$\$ (\$6 - \$12)                             
\end{tabularx}
\caption{Vergleich ESP8266 zu ESP32.}
\label{tab:ESP}
\end{table}

Wichtig ist, dass ein ESP ausgewählt wird, welches einen Anschluss für eine abgesetzte Antenne hat, da die Leiterplatte im Gehäuse verbaut wird. Dazu eignet sich der Espressif ESP32-32U. Dieser ist in Abbildung \ref{fig:Produktbild_ESP32_32U_Wroom} als Wroom dargestellt und in Abbildung \ref{fig:Produktbild_ESP32_32U_DevKit} als Development Kit.

\begin{figure}[!h]
\center
\includegraphics[width = 0.4\textwidth]{graphics/Produktbild_ESP32}
\caption{ESP32-32U Wroom.}
\label{fig:Produktbild_USB_UART_ESP}
\end{figure}

\begin{figure}[!h]
\center
\includegraphics[width = 0.4\textwidth]{graphics/Produktbild_ESP32_2}
\caption{ESP32-32U DevKit.}
\label{fig:Produktbild_ESP32_32U_DevKit}
\end{figure}