\subsection{Förderband}
\label{subsubsec:Foerderband}

Da gemäss Kapitel \ref{subsec:Entscheidungfindung_des_Aufbaus} ein Förderband eingesetzt wird um das Glas zu bewegen, muss ein Motor eingesetzt werden, welcher das Glas mitsamt Inhalt transportieren kann. Ausserdem sollen die verschiedenen Anfahrtspositionen exakt angefahren werden können. 

\subsubsection{Anforderungen}\label{par:Anforderungen_Foerderband}

\begin{tabularx}{\textwidth}{lllX}
Mechanischer Aufbau & : & Länge & $80\pm10cm$ (Unter Annahme 10cm pro Flasche)\\
 & & Geschwindigkeit & 180cm pro 30s = 6cm/sek \\
 & & Belastbarkeit & 9.81N (1kg) \\
 & & Oberfläche & Rutschfest \\
 & & Führung & zwei Führungsstangen um den Schlitten zu bewegen. \\
 & & Schlitten & evt. 3D Druck \\
\end{tabularx}

\subsubsection{Verwendeter Aufbau}\label{par:Verwendeter_Aufbau}

Der Aufbau der Cocktailmaschine soll gemäss der Entscheidungsfindung des Aufbaus der Cocktailmaschine \ref{subsec:Entscheidungfindung_des_Aufbaus} ähnlich aussehen, wie bei der CocktailAvenue \ref{subsubsec:Aufbau_CocktailAvenue}. Allerdings sollen die Getränkebehälter nicht sichtbar hinter der Maschine platziert werden. Somit könnte man sagen, dass es eine Mischung aus der CocktailAvenue \ref{subsubsec:Aufbau_CocktailAvenue} und dem Cocktailmixer \ref{subsubsec:Aufbau_Der_Cocktailmixer} sein wird. Ein erster Prototyp soll zuerst aus Holz angefertigt werden, bevor die Maschine dann aus Metall (vorzugsweise Chromstahl) aufgebaut wird.