\section{ESP32}\label{Appendix:ESP_32}

\subsection{Tabellarischer Vergleich zwischen ESP32 und ESP8266}\label{Appendix:ESP32_vs_ESP8266}

\begin{table}[H]
\center
\begin{tabular}{|l|c|c|}
\hline
\textbf{MCU}                    & \textbf{Xtensa Single-Core} & \textbf{Xtensa Dual-Core} \\

			                   & \textbf{32-bit L106 (ESP8266)} & \textbf{32-bit LX6 (ESP32)} \\ \hline

802.11 b/g/n     		& HT20                           & HT40                                       \\
Bluetooth              	& No                             & Bluetooth 4.2 and BLE                      \\
Arbeitsfrequenz			& 80 MHz                         & 160 MHz                                    \\
SRAM                   	& No                             & Yes                                         \\
Flash                  	& No                             & Yes                                          \\
GPIO                   	& 17                             & 36                                         \\
SPI/I2C/I2S/UART       	& 2/1/2/2                        & 4/2/2/2                                    \\
ADC                    	& 10-bit                         & 12-bit                                     \\
Ethernet Interface 		& No                             & Yes                                          \\
Touchsensor           	& No                             & Yes                                          \\
Temperatursensor     	& No                             & Yes                                          \\
Hall-Sensor     			& No                             & Yes \\
Arbeitstemperatur    	& -40ºC to 125ºC                 & -40ºC to 125ºC                             \\
Price                  	& \$ (3\$ - \$6)                 & \$\$ (\$6 - \$12)    \\                         
\hline
\end{tabular}
\caption{Vergleich ESP8266 zu ESP32.}
\label{tab:ESP}
\end{table}

\subsection{Strapping-Pins}\label{Appendix:ESP32_Strapping}

\begin{table}[H]
\center
\begin{tabular}{|c|c|c|c|c|c|c|}
\hline
\multicolumn{7}{|c|}{\textbf{Ausgangsspannung internen Spannungsregler (VDD\_SDIO)}}\\
\hline
RS232 & ESP & default & \multicolumn{2}{|c|}{\textbf{3.3V}} & \multicolumn{2}{|c|}{\textbf{1.8V}}\\
\hline
\sout{RTS} & \sout{IO12} & Pull-down & \multicolumn{2}{|c|}{\sout{\textcolor{red}{0}}\footnotemark} & \multicolumn{2}{|c|}{\sout{1}}\\
\hline
\multicolumn{7}{|c|}{\textbf{Boot-Modus}}\\
\hline
RS232 & ESP & default & \multicolumn{2}{|c|}{\textbf{SPI-flash Boot}} & \multicolumn{2}{|c|}{\textbf{Download Boot}}\\
\hline
DTR & IO0 & Pull-up & \multicolumn{2}{|c|}{1} & \multicolumn{2}{|c|}{0}\\
\hline
- & IO2 & Pull-down & \multicolumn{2}{|c|}{Egal} & \multicolumn{2}{|c|}{0}\\
\hline
\multicolumn{7}{|c|}{\textbf{Debugging Log Print über U0TXD während Booten}}\\
\hline
RS232 & ESP & default & \multicolumn{2}{|c|}{\textbf{U0TXD Active}} & \multicolumn{2}{|c|}{\textbf{U0TXD Silent}}\\
\hline
RTS & IO12 & Pull-down & \multicolumn{2}{|c|}{\textcolor{red}{1}} & \multicolumn{2}{|c|}{0}\\
\hline
\multicolumn{7}{|c|}{\textbf{Timing\footnotemark des SDIO}}\\
\hline
RS232 & ESP & default & \shortstack{FF Sampling \\ FF Output} & \shortstack{FF Sampling \\ SF Output} & \shortstack{SF Sampling \\ FF Output} & \shortstack{SF Sampling \\ SF Output} \\
\hline
CTS & IO15 & Pull-up & 0 & 0 & 1 & 1 \\
\hline
- & GPIO5 & Pull-up & 0 & 1 & 0 & 1 \\
\hline
\end{tabular}

\caption{Tabelle Pinkonfiguration für Strapping-Pins.}
\label{tab:Strapping_pins}
\end{table}

\newpage

\textbf{Spannung des internen Spannungsgeglers (VDD\_SDIO)}

Das ESP32 hat einen eingebauten host controller für SD/SDIO/MMC-Speichergeräte. Dieser kann mit 3.3V (IO12 = 0) oder 1.8V (IO12 = 1) betrieben werden. Wird der Pin auf 1 gesetzt, könnte ein Brown-out der IO-Versorgungspannung VCC\_IO ausgelöst werden.

\textbf{Booting Mode}

Der Pin IO0 gibt vor, ob das ESP32 vom internen SPI-flash bootet (IO0 = 1) oder ob neuer Code in den flash Speicher geschrieben wird (IO0 = 0). Soll das ESP vom Flash-Speicher booten, so hat der Pin IO2 kein Einfluss, um jedoch den Code speichern zu können, muss der Pin auf 0 sein.

\textbf{De-/Aktivieren vom Debug-Log über U0TXD während dem Bootvorgang}

Über den Pin IO12 kann konfiguriert werden, ob während dem Booten ein Debug-Log über die Serielleschnittstelle gesendet wird (IO12 = 1) oder nicht (IO12 = 0).

\textbf{Timing der Kommunikation mit dem SDIO Slave}:

Über die Pins IO15 und IO5 kann das Übertragungsprotokoll des SDIO-Slaves festgelegt werden. Dabei kommt es darauf an, ob das Sampling auf eine fallende Flanke (IO15 = 0) oder steigende Flanke (IO15 = 1) geschehen soll, und ob der Output auf eine fallende Flanke (IO5 = 0) oder steigende Flanke (IO5 = 1) geschehen soll.

\footnotetext[18]{Da das ESP32-WROOM-32U einen 3.3V SPI flash integriert hat, kann der MTDI nicht auf 1 gesetzt werden, wenn die Module aufgestartet sind.}
\footnotetext{FF = Fallende Flanke, SF = Steigende Flanke}
