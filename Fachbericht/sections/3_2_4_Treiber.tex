\subsubsection{Treiber}\label{subsec:Treiber}

Der Treiber, welcher vom Mikrocontroller angesteuert wird und den Motor ansteuert, wurde wie der Motorentyp vom betreuenden Dozent vorgeschlagen. Es handelt sich beim Treiber um einen TMC4671. Im folgenden wird ermittelt, ob der Treiber zum Motor passt.

\paragraph{TMC4671}\label{par:Anforderungen_TMC4671}\mbox{}\\

Im folgenden soll aufgezeitg werden, was der TMC4671 alles kann und wofür er gemacht wurde. In der linken Spalte kann das Vergleichskriterium entnommen werden, auf der rechten Seite ist jeweils der obere Textblock eine Beschreibung dessen, was der Treiber kann (\textbullet). Im Textblock darunter kann entnommen werden, wie der Treiber genau zum Motor steht und ob die Konfiguration passt ($\square$).

\begin{tabbing}
\parbox[t]{.25\textwidth}{Motorentypen} \= \parbox[t]{.75\textwidth}{\textbullet Der Treiber ist in der Lage drei verschiedene Motorentypen anzusteuern.
Diese umfassen Stepper-Motoren, normale DC-Motoren und Brushless-DC-Motoren.\\

$\square$Der AKM 22H ist ein BLDC-Motor und wird folglich vom Treiber unterstützt.}\\
\\
\parbox[t]{.25\textwidth}{Encoder Treiber} \> \parbox[t]{.75\textwidth}{\textbullet Weiter kann der  Treiber verschiedene Encodersysteme decodieren.
Diese sind Digitale-/Analoge- und Hall-Encoder.\\

$\square$ Der AKM 22H hat einen eingebauten Resolver. Der Resolver ist ein Analoger Encoder, welcher die Position des Rotors mit zwei Spulen ermittelt. Die Position wird in einen Sin- und einen Cos-Anteil zerlegt und macht so die genaue Lage des Rotors für den Treiber messbar.}\\
\\
\parbox[t]{.25\textwidth}{Kommunikation\newline mit CPU} \>\parbox[t]{.75\textwidth}{\textbullet Die Kommunikation mit dem CPU kann über zwei Kommunikationsprotokolle geschehen, nämlich SPI und UART. Weiter hat es einen Step/Dir Anschluss für einen Stepper-Motor, welcher für dieses Projekt jedoch nicht verwendet wird. Auch ein Status-Pin ist vorhanden, womit der Mikrocontroller über den Zustand des Treibers informiert wird.\\

$\square$ Die Kommunikation mit dem CPU hat nur mit dem Treiber zu tun und ist nicht vom Motor abhängig.}\\
\\
\parbox[t]{.25\textwidth}{Sonstiges} \>\parbox[t]{.75\textwidth}{\textbullet Der Strom, welcher durch die Spulen fliesst, kann zusätzlich gemessen werden. Um die Spulen jedoch richtig ansteuern zu können, benötigt es einen Gatedriver.\\

$\square$ Die Strommessung ist für den Motor belanglos, die Motordaten geben jedoch die Dimensionierung der Gate-Treiber bzw. MOSFET's vor.\\

\textbullet Als Referenzpunkte wie z.B eine Home-Position gibt es Referenz-Eingänge.\\

$\square$ Die Referenzpunkte sind nur für den Treiber von Bedeutung und beeinflussen die Auswahl des Motors nicht.}\\
\\
\end{tabbing}

\paragraph{TMC6200}\label{par:Anforderungen_TMC6200}\mbox{}\\

\todo{Gate-Treiber beschreiben}