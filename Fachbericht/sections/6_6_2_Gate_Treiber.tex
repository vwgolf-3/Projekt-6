\subsubsection{Gate-Treiber}
\label{subsubsec:Inbetriebnahme_Gate_Treiber}

Der Gate-Treiber wird ebenfalls über die SPI-Schnittstelle in Betrieb genommen. Dazu werden die Parameter verwendet, welche in der TMCL-IDE verwendet wurden. Eine detaillierte Auflistung der beschriebenen Register ist im Anhang Kapitel \ref{Appendix:TMC6200_Register} zu finden. Mit den Standardparametern muss am Motoranschluss die selbe Signalfolge anliegen wie am sie auch am FOC-Treiber anliegen (Verstärkt durch H-Brücke). Die Initialisierung sowie das Auslesen gewisser Register ist mit der Testapplikation ''\textit{3\underline{ }Motor\underline{ }Openloop}'' möglich.

Das Setup, welches in Kapitel \ref{subsubsec:Inbetriebnahme_FOC_Treiber} erwähnt wurde, ist jetzt um einen Baustein erweitert worden und in Anhang Kapitel \ref{Appendix:TMC4671_Setup} ersichtlich.

Vorgehen:
\begin{enumerate}
\item Benötigte Applikation aus dem Software-Ordner auf dem USB-Stick in Atmel Studio öffnen.\\
\textcolor{magenta}{Software\textrightarrow Atmega\textrightarrow 3\underline{ }Motor\underline{ }Openloop\textrightarrow 1\underline{ }Motor\underline{ }Testsoftware\textrightarrow Motor}\\


\item Software anpassen:\\
\textcolor{OliveGreen}{
	initTMC6200;\\
	initTMC4671\underline{ }Openloop();\\
\\
    while (1) \\
    \{\\
		\underline{ }delay\underline{ }ms(5000);\\
		read\underline{ }registers\underline{ }TMC6200();\\
		\underline{ }delay\underline{ }ms(10000);\\
		read\underline{ }registers\underline{ }TMC4671();\\
    \}
}\newline
\item Software hochladen:\\
\textcolor{blue}{AtmelStudio\textrightarrow Tools\textrightarrow PartyMixer}\\

\item SPI-Kommunikation und Signale an der H-Brücke überprüfen. Im Anhang Kapitel \ref{Appendix:TMC6200_SPI} sind die Messbilder zur SPI-Kommunikation zu finden und in Anhang \ref{Appendix:TMC6200_Gate_Ctrl} die Bilder zur Gate-Ctrl vom TMC6200 zur H-Brücke.

\end{enumerate}