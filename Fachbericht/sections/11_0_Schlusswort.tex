\section{Persönliches Schlusswort}
\label{sec:Persönliches Schlusswort}

Die ausführliche Recherchearbeit, welche im \textcolor{red}{\textbf{Fachbericht 5}} getätigt wurde, zahlte sich im Projekt 6 aus. Durch diese Recherche war es möglich viele Systeme sauber und relativ schnell einzubinden. Es wurde uns jedoch erst später bewusst, welchen Aufwand wir auf uns genommen hatten und was es bedeutet ein voll umfängliches Gerät mit Elektronik, Software und Mechanik zu konstruieren. Schlussendlich hatten wir einiges mehr als doppelt so viel Zeit und Arbeit hineingesteckt, als es für eine Bachelor Thesis geplant wäre. Dies ist jedoch nicht nur der Arbeit selbst zu verschulden, sondern auch weil wir uns das Ziel gesetzt hatten möglichst sauber und penibel zu arbeiten. Besonders die Entwicklung der Software und der Bedienmenüs, sowie der Mechanik nahmen sehr viel Zeit in Beanspruchung. Dafür wurden jedoch auch viele Extras eingebunden und es wurde über das eigentliche Ziel hinausgeschossen. Wir haben nicht nur alle Pflicht- und fast alle Wunschziele erreicht, sonder auch noch einiges mehr. Es wurde geschafft eine voll funktionsfähige Maschine zu entwickeln mit coolen Features. Daher denken wir, dass dies  ein Punkt ist, auf den wir auch ein wenig stolz sein können. 

Doch auch mit viel Frust mussten wir klarkommen. Der Motor funktionierte bis kurz vor Ende des Projektes nicht wirklich und wir spielten schon länger mit dem Gedanken den Motor vollkommen aufzugeben. Dies schaffte viel Unmut und wir mussten auch lernen miteinander in schwierigen Situationen umzugehen. Unsere Hartnäckigkeit zahlte sich jedoch am Schluss aus. 