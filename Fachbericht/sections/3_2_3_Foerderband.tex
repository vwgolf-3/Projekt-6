\subsubsection{Förderband}
\label{subsubsec:Foerderband}

Gemäss Kapitel \ref{subsubsec:Entscheidungfindung_des_Aufbaus} ein Förderband eingesetzt wird um das Glas zu bewegen, muss ein Motor eingesetzt werden, welcher Das Glas mitsamt Inhalt Transportieren kann. Ausserdem sollen die verschiedenen Anfahrtspositionen exakt angefahren werden können. Dazu werden verschiedene Motorentypen miteinander verglichen.

\paragraph{Anforderungen}\label{par:Anforderungen_Foerderband}

\begin{itemize}
\item Mechanischer Aufbau
\begin{itemize}
\item Länge: $80\pm10cm$ (Unter Annahme 10cm pro Flasche)
\item Geschwindigkeit: 180cm pro 30s = 6cm/sek
\item Belastbarkeit: 9.81N (1kg)
\item Oberfläche: Rutschfest
\item Führung: zwei Führungsstangen mit Gewindestange um den Schlitten zu bewegen.
\item Schlitten: 3D Druck
\end{itemize}
\end{itemize}

\paragraph{Verwendeter Aufbau}\label{par:Verwendeter_Aufbau}

