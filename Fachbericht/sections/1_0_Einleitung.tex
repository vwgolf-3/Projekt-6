\clearpage
\section{Einleitung}
\label{sec:Einleitung}

Eine gelungene Party auf die Beine zu stellen verlangt einem einiges ab. Vor allem kostet es eine Menge Aufwand und Zeit. Dies gilt besonders, wenn es darum geht mit vielen Freunden zusammen zu feiern. Neben der gelungenen Musikauswahl und den Snacks darf eines auf gar keinen Fall fehlen, die Getränke. Um diese sicherzustellen, gibt es mehrere Möglichkeiten. Einerseits könnte jeder seine eigenen Getränke mitbringen, was jedoch bedeutet, dass es unter Umständen eine riesige Sauerei gibt oder viele Flaschen in der Gegend rumstehen. Anderseits könnte man als Gastgeber selber anbieten Cocktails zu mixen und so den Getränkenachschub zu gewährleisten. Da gibt es jedoch ein grosses Problem. Denn wären wir die Gastgeber, so würden wir nicht den ganzen Abend hinter der Bar stehen wollen, sondern lieber bedenkenlos mitfeiern. Damit genau dies möglich ist haben wir uns in diesem und dem nächsten Projekt (5\&6) dazu entschieden eine automatisierte Cocktailmaschine zu entwerfen. Diese soll vollkommen autonom arbeiten und sollte problemlos von jeder beliebigen Person und in fast jedem Zustand bedient werden können. 

In den folgenden Kapiteln ist dokumentiert, wie die Cocktailmaschine aussehen soll und aus welchen Teilsystemen diese bestehen wird. Ausserdem werden die einzelnen Teilsysteme genauer unter die Lupe genommen und in einem systemspezifischen Testverfahren evaluiert. Dieses Projekt bietet demnach die Basis des Projekt 6 und soll dieses so gut wie möglich vorbereiten.