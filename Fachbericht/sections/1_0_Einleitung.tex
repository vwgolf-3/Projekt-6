\clearpage
\section{Einleitung}
\label{sec:Einleitung}

Eine gelungene Party auf die Beine zu stellen verlangt einem einiges ab. Vor allem kostet es eine Menge Aufwand und Zeit. Dies gilt besonders, wenn es darum geht mit vielen Freunden zusammen zu feiern. Neben der gelungenen Musikauswahl und den Snacks darf eines auf gar keinen Fall fehlen, die Getränke. Um diese sicherzustellen, gibt es mehrere Möglichkeiten. Einerseits könnte jeder seine eigenen Getränke mitbringen, was jedoch bedeutet, dass es unter Umständen eine riesige Sauerei gibt oder viele Flaschen in der Gegend rumstehen. Anderseits könnte man als Gastgeber selber anbieten Cocktails zu mixen und so den Getränkenachschub zu gewährleisten. Da gibt es jedoch ein grosses Problem. Als Gastgeber möchte man nicht den ganzen Abend hinter der Bar stehen müssen, sondern lieber bedenkenlos mitfeiern. Damit genau dies möglich ist haben wir uns dazu entschieden eine automatisierte Cocktailmaschine zu entwerfen. Diese soll vollkommen autonom arbeiten und sollte problemlos von jeder beliebigen Person und in fast jedem Zustand bedient werden können. 

In den folgenden Kapiteln ist dokumentiert, wie die Cocktailmaschine im Detail aussieht. Dazu gehören die elektronischen Teilsysteme, das dazugehörige Printdesign, die Software und die Mechanik.