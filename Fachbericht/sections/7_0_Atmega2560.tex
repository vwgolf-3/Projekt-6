\subsection{Atmega2560}
\label{subsec:Software_Atmega2560}

Die Software für den Atmega2560 ist wie folgt aufgebaut: Im \textbf{Init/Main} werden die Variabeln deklariert, welche im Programm benutzt werden, um die aus den Kommunikationsbuffern geholten Daten zur Verarbeitung zu Speichern. Zudem werden hier die Adressen auf die Funktionen deklariert, welche die Module (z.B IO, Speicher, Devices, Interfaces) initialisieren und die Buffer in der main-Schleife prüfen. Durch Aufrufen des h-Files ''Cocktail\_Statemachine.h'' werden die Adressen der projektspezifischen Funktionen deklariert und somit die \textbf{User-Applikation} eingebunden. Darunter befinden sich die Libraries, welche die Registernummern oder Befehlssätze beinhalten, um die Devices anzusteuern. Sie bilden die Schnittstelle zwischen User-Applikation und Kommunikationsinterface. Dies ist vergleichbar mit einem \textbf{Application-Programming-Interface}\footnote{API, Schnittstelle zur Anwendungsprogrammierung}, einem Programmteil, welcher eine Verbindung eines Programms zu einem anderen Programm ermöglicht. Die Informationen werden dabei standardisiert zwischen den Anwendungen ausgetauscht. Die Daten oder Befehle werden strukturiert nach einem definierten Syntax übergeben\todo{cite: https://www.datacenter-insider.de/was-ist-ein-application-programming-interface-api-a-735797/}. Der Zugriff auf die Hardware des Mikrocontrollers (SPI- und UART-Interface) erfolgt mittels den AVR-Registern. Die Libraries, die dafür verwendet werden befinden sich folglich zwischen dem ''API''-Layer und der Hardware und können mit der \textbf{Hardware Abstraction Layer}\footnote{HAL, Hardwareabstraktionsschicht} verglichen werden. Es wird nur über diese Funktionen auf die entsprechende Hardware zugegriffen.