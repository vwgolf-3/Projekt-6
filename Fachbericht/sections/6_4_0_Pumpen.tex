\subsubsection{Pumpen}
\label{subsubsec:Inbetriebnahme_Pumpen}

Um die Flüssigkeitsbeförderung zu testen, wurde ein Testprogramm erstellt, welches die 12 Pumpen der Reihe nach für eine kurze Zeitdauer einschaltet. Dies funktionierte auf Anhieb und es konnten einige Tests durchgeführt werden. 

Einer der wichtigsten Tests war es dabei, den maximalen Durchfluss der Pumpen zu bestimmen. Dies ermöglichte eine Abschätzung, wie lange es benötigt um einen Cocktail herzustellen. Um das Pflichtziel von unter einer Minute erfüllen zu können, müssen daher die Pumpen genügend schnell Pumpen können.

Im Schnitt ergab sich dabei eine Abfüllzeit von 26 Sekunden für eine Menge von 5dl. Erstaunlich ist jedoch, dass die Pumpen unterschiedlich lange benötigen für die selbe Menge. Somit benötigte die langsamste Pumpe 28 Sekunden, wobei die schnellste Pumpe lediglich 24 Sekunden benötigte. Das Zusammenspiel von Pumpen und Motor wird in der Zielerreichung in Kapitel \ref{sec:Zielerreichung} aufgezeigt.