\subsubsection{Pumpen}
\label{subsubsec:Inbetriebnahme_Pumpen}

Um die Flüssigkeitsbeförderung zu testen, wurde ein Testprogramm erstellt, welches die 12 Pumpen der Reihe nach für eine kurze Zeitdauer einschaltet. Dies funktionierte auf Anhieb und es konnten einige Tests durchgeführt werden. 

Einer der wichtigsten Tests war es dabei, den maximalen Durchfluss der Pumpen zu bestimmen. Dies ermöglichte eine Abschätzung, wie lange es benötigt um einen Cocktail herzustellen. 

In der folgenden Tabelle sind die durchschnittlichen Zeiten aus jeweils 5 Testläufen der einzelnen Pumpen zu sehen. Da es sich nicht um ein System handelt, welches hoch präzise Ansprüche zu erfüllen hat, wurde auf Sekunden gerundet.

\todo{Robin Zeitmessung der einzelnen Pumpen einfügen, wie lange es benötigt um 5dl abzufüllen}