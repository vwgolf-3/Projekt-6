\subsubsection{RFID}
\label{subsubsec:Inbetriebnahme_RFID}

Beim RFID-Leser handelt es sich wie in Kapitel \ref{subsubsec:RFID} um das MFRC522 Evaluierungsboard (Breakout-Board). Dieses Kommuniziert über SPI mit einem gewünschten Controller. Die Pinbelegung dazu ist in Abbildung \ref{fig:MFRC522} zu sehen. Der grosse Vorteil dieses Evaluirungsboard ist es, dass es dazu eine Arduino Library gibt mit vielen Beispielen.

Um den Leser in Betrieb nehmen zu können, wurde dieser wie folgt angeschlossen:


\begin{itemize}
\item Vcc: An 3.3V vom ESP32
\item GND: An GND vom ESP32
\item SDA: An IO5 vom ESP32
\item SCK: An IO18 vom ESP32
\item MISO: AN IO19 vom ESP32
\item Reset: AN IO22 vom ESP32
\item MOSI: AN IO23 vom ESP32
\end{itemize}

\begin{figure}[h!]
\center
\includegraphics[width = 0.58\textwidth]{graphics/MFRC522}
\caption{Anschauungsbild MFRC522 Evaluierungsboard \cite{nxp_bv_2010_antenna_2010}}
\label{fig:MFRC522}
\end{figure}

Diese Pinbelegung ist von der Arduino-Library gegeben und wird auch so bei der Cocktailmaschine eingesetzt, da es keinen Grund gibt diese direkt in der Library zu ändern. Lediglich die beiden Pin's \flqq SDA\frqq~und \flqq Reset\frqq~können flexibel festgelegt werden.

Um das Lesegerät in Betrieb nehmen zu können, wurden nach dem Anschliessen des ESP32 an den Computer folgende Schritte unternommen: 

\begin{enumerate}
\item Einbinden der MFRC522 Library unter: \textcolor{blue}{Werkzeuge \textrightarrow Bibliotheken verwalten} \newline
\item Öffnen des Beispiels DunmInfo unter: \textcolor{blue}{Datei \textrightarrow Beispiele \textrightarrow MFRC522 \textrightarrow DumpInfo} \newline
\item Einstellen des Reset und des SDA (SS) Pin's in den defines des Codes\newline
\item Einbinden des ESP32-Boards unter: \textcolor{blue}{Werkzeuge \textrightarrow Board \textrightarrow Boardverwalter \textrightarrow esp32} \newline
\item Einstellen des verwendeten Boards unter: \textcolor{blue}{Werkzeuge \textrightarrow Board \textrightarrow ESP32 Arduino \textrightarrow ESP32 Dev Module} \newline
\item Einstellen des richtigen COM-Ports unter: \textcolor{blue}{Werkzeuge \textrightarrow Port} (Kann im Geräte-Manager des Computers nachgeschaut werden) \newline
\item Upload des Pogrammes mittels Upload-Button \newline
\item Öffnen des Serial Monitors unter: \textcolor{blue}{Werkzeuge \textrightarrow  Serieller Monitor} \newline
\end{enumerate}

Nun konnten die verschiedenen RFID-Tag's eingelesen werden und  die gelesenen Informationen wurden am seriellen Monitor ausgegeben. \cite{pcbreflux_esp32_2017}




