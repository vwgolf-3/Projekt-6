\subsubsection{5V Speisung}
\label{subsubsec:5V Speisung}

Der Mikrocontroller, sowie die Durchflussmessgeräte und das Display  werden mit 5V betrieben. Aus diesem Grund wurde eine 5V Speisung implementiert. Dazu wird derselbe Schaltspannungsregler wie bei der 12V Speisung in Kapitel \ref{subsubsec:12V Speisung} verwendet, jedoch mit anderen Komponenten. Die Realisierung der 5V Speisung kann in Abbildung \ref{fig:Schema_Speisung_5V} eingesehen werden.\\

\paragraph{Schema}\mbox{}

Das Schema in Abbildung \ref{fig:Schema_Speisung_5V} ist wie bei der 12V Speisung gemäss Kapitel \ref{subsubsec:12V Speisung}, in fünf Teile unterteilt. Da sind zuerst die Eingangskondensatoren, welche mit C31, C33 \& C35 realisiert sind. Diese Entstörstufe wird wiederum gefolgt von einem Spannungsteiler, welcher den Enable auf aktiv setzt. Der eigentliche Regler wird hier mittels des IC6, D5 \& L4 realisiert. Mit zwei Spannungsteiler wird die gewünschte Ausgangsspannung sowie die \flqq Overvoltage-Protection\frqq~ eingestellt. Vor dem Ausgang der Schaltung ist erneut eine Kondensatorstufe implementiert, welche das Ausgangssignal glättet.

\begin{figure}[h!]
	\centering
	\includegraphics[width=\textwidth]{graphics/Schema_Speisung_5V.png}
	\caption{Schema der 5V-Speisung}
	\label{fig:Schema_Speisung_5V}
\end{figure} 

\paragraph{Funktionsbeschrieb der Schaltung}\mbox{}

Bei der 5V-Speisung wurde mittels R29 \& R31 ein Spannungsteiler realisiert, welcher das IC gemäss Kapitel \ref{subsubsec:12V Speisung} auf aktiv setzt.

Das Widerstandsverhältnis von R41 \& R42, das die Ausgangsspannung definiert, ist gemäss Formel \ref{equ:Ausgangsspannung_12V} berechnet. Daraus resultiert für R41=301k$\Omega$ und für R42=57.6k$\Omega$, was einer Ausgangsspannung von 4.98V entspricht. Die Widerstände sollten in der Grössenordnung > 10k$\Omega$ gewählt werden, um unnötige Verluste zu vermeiden.  

Beim Überspannungsschutz ist darauf geachtet worden, dass der Mikrocontroller nur in einem Spannungsbereich von 4.5V-5.5V betrieben werden darf. Die maximal verträgliche Eingangsspannung liegt laut Datenblatt bei 6V. Somit muss der Überspannungsschutz so gestaltet werden, dass die Schwelle von 6V nicht überschritten werden kann. Um dies zu erreichen, ist für R36=301k$\Omega$ und R37=53k$\Omega$ gewählt worden. Gemäss Formel \ref{equ:Vovp_12V} erhält man so eine Überspannungsschutzschwelle von 6V \cite[S.1]{atmel_atmel_2014}.

Der interne Oszillator läuft wiederum bei einer Frequenz von 100kHz. Bei der ausgewählten Spule L4 von 47$\mu$H erhält man mittels Formel \ref{equ:12V_Spulenberechnung} ein $\Delta$I$_{L}$ von 0.953A. Hier gilt gemäss Datenblatt, dass die gewählte Spule auf mindestens 125\% des maximalen Ausgangsstroms von 3A ausgelegt werden muss. Auch der Gleichstromwiederstand der Spule sollte $ \leq \ $ 200m$\Omega$  sein \cite[S.3]{monolithic_power_systems_mp24943_2011}. 

Mit den Kondensatoren C46, C48 \& C50 wird die Ausgangsspannung zum Abschluss noch geglättet. Bei den Eingangskondensatoren sowie den Ausgangskondensatoren sollte es sich gemäss Datenblatt um low ESR Typen handeln, um eine bessere Störungsunterdrückung zu gewährleisten. Zum Abschluss ist ein Ferrit implementiert worden, welcher allfällige hochfrequente Störungen herausfiltert.

Bei der 5V Speisung wurde ebenfalls ein Jumper zu Testzwecken und zwei LED's implementiert. Ausserdem findet sich auch hier wieder ein Ferrit FL3, welcher letzte Störungen beseitigt. 


