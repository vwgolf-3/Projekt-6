\subsubsection{Durchflussmessgeräte}
\label{subsubsec:Inbetriebnahme_Durchflussmessgeräte}

Die Durchflussmessgeräte in Betrieb zu nehmen gestaltete sich ein wenig schwieriger als bei den Pumpen in Kapitel \ref{subsubsec:Inbetriebnahme_Pumpen}. Da die Durchflussmessgeräte bei Durchfluss Pulse ausgeben, müssen die positiven Flanken gezählt werden. Dafür wurde das Testprogramm für die Pumpen erweitert. Auch dies funktionierte auf Anhieb und es konnte mit ersten Testläufen begonnen werden. 

Da die Durchflussmessgeräte auf dem volumetrischen Prinzip basieren, spielt es keine Rolle ob die 12 Pumpe exakt gleich stark Pumpen oder nicht. Es wurde jedoch schnell festgestellt, dass die Durchflussmessgeräte kleine Toleranzen zueinander aufweisen, jedoch für sich selber relativ exakt arbeiten. Dies hatte zur Folge, dass die einzelnen Durchflussmessgeräte jeweils separat mit der Software kalibriert werden mussten. Dabei ist aufgefallen, dass die Anzahl der Pulse im Verhältnis zum Volumen für die einzelnen Durchflussmessgeräte auch nicht ganz linear ist. Somit wurden zwei Kalibrierungen pro Durchflussmessgerät vorgenommen. Einmal wurden die Durchflussmessgeräte auf 3dl und einmal auf 5dl kalibriert. Dies wurde aus dem einfachen Grund gemacht, dass die Getränke in diesen zwei Grössen hergestellt werden. Um dies sicherstellen zu können, wurde jedes der 12 Durchflussmessgeräte in jeweils 6 Durchgängen für 3dl und 5dl kalibriert und Softwaremässig erfasst. Dabei wurde das abgefüllte Gewicht von Wasser mit einer Küchenwaage gemessen. Wasser hat dabei den Vorteil, dass es die Dichte von nahezu 1000kg/m$^3$ besitzt, was so viel bedeutet, dass 1g Wasser 1ml entspricht. \cite{wagner_iapws_2002}


Als sichergestellt wurde, dass die Durchflussmessgeräte kalibriert sind, wurde mit der eigentlichen Testreihe begonnen. Es ging dabei um die Abfüllgenauigkeit der einzelnen Pumpstationen. Besser gesagt um die Beständigkeit. Dazu wurden erneut einige Testreihen aufgesetzt. Einerseits wurde für jede Pumpe 10 Mal 3dl und 10 Mal 5dl gepumpt und geschaut, wie gross die Toleranz ist. In einem weiteren Schritt wurde dann das Zusammenspiel der einzelnen Pumpen gemessen. 

\textbf{Test 1:} \\
In diesem Test wurden alle 12 Pumpen 10x Durchgemessen. Dabei wurde für jede Pumpe 10x 3dl abgefüllt und mit einer Küchenwaage gemessen, wie sehr das Ergebnis schwankt. Die grösste festgestellte Toleranz lag hierbei bei 2g Abweichung, was erstaunlich ist, da dies einer Genauigkeit von 0.7\% entspricht.

\textbf{Test 2:} \\
Das Selbe wie in Test 1 wurde auch in Test 2 durchgeführt. Allerdings wurde die Abfüllmenge auf 5dl erhöht. Hierbei ergab sich eine grösste Abweichung von 5g, was einer Genauigkeit von 1\% entspricht.

\textbf{Test 3:} \\
Im letzten Test wurden noch einmal 5dl abgefüllt, jedoch nicht von einer einzigen Pumpe sondern von jeder Pumpe ein zwölftel, wie es bei der Cocktailerstellung der Fall ist. Dies ist der ''Worst-Case''. In diesem Test wurde eine maximale Toleranz bei 10 Durchläufen von 18g gemessen, was 3.6\% entspricht.

Diese Testreihe ist für die Zielerreichung in Kapitel \ref{sec:Zielerreichung} essentiell.