\newpage
\section{Software}
\label{sec:Software}

Im Folgenden Kapitel werden die beiden Softwareteile beschrieben. Die Software für den Mikrocontroller beinhaltet die komplette Ansteuerung der Teilsysteme, die Führung durch das Menu auf dem Display sowie die Ausführung der Funktionen bei Auswahl auf dem Display. Ausserdem wird hier der gesamte Stand der Maschine gespeichert.

Die Software für das ESP32 erweitert die Funktionalität der Maschine, indem einige Funktionen auch über eine Android-Applikation auf dem Handy aufgerufen werden können. Die Software auf dem ESP32 übernimmt dabei eine Zwischenfunktion, indem es die eingehende Bluetooth-Kommunikation der Android-Applikation übersetzt und die Daten an den Mikrocontroller sendet. Ausserdem werden Daten vom Mikrocontroller abgefragt und auch das RFID-System wird in das ESP32 implementiert.