\subsubsection{BLDC und H-Brücke}
\label{subsubsec:Inbetriebnahme_BLDC_und_H-Brücke}


Die H-Brücke wird mit dem selben Setup und der selben Software wie in Kapitel \ref{subsubsec:Inbetriebnahme_Gate_Treiber} gemacht. Es komm. Der Vorgang ist auch derselbe, weshalb dieser weggelassen wird. Bei der H-Brücke geht es darum, die Schaltsignale der FETs zu überprüfen, bevor der Motor angeschlossen wird. Sind die Signale gut, kann der Motor angeschlossen werden.

Das Setup mit dem Motor ist im Anhang Kapitel \ref{Appendix:H_Bruecke_Setup} und die Schaltsignale im Anhang Kapitel \ref{Appendix:H_Bruecke_Schaltsignale} ersichtlich.

Wird jetzt der Motor angeschlossen, dreht sich dieser mit der vorgegebenen Geschwindigkeit.