\clearpage
\subsection{Motor}
\label{subsec:Motor}

Um ein Glas während der Zubereitung hin und her zu bewegen, wird eine Antriebsgruppe benötigt. Die Auswahl der Motorengruppe wurde mit dem Dozenten ausgewählt. Der Entscheid fiel dabei auf den Brushless DC-Motor AKM22h von Sigmatec. Auch dessen Ansteuerung ergab sich durch die schon vorhandenen EVAL-Boards mit TMC4671 und UPS 10A70V. Im Projekt 6 wird anstelle des UPS 10A70V ein TMC6200 verwendet. Das benötigte Feedback über die Lage des Rotors wird vom ABN-Encoder AMT33 von CUI devices geliefert. Zwei Shunts geben Auskunft über die Bestromung der Spulen. Auf die erwähnten Komponenten wird im Folgenden eingegangen. Abbildung \ref{fig:Blockdiagramm_TMC4671_und_TMC6200} zeigt, wie die Komponenten zusammenhängen.

\begin{figure}[H]
	\centering
	\includegraphics[width=0.8\textwidth]{graphics/Blockdiagramm_TMC4671_und_TMC6200}
	\caption{Blockschaltbild Konfiguration IC's mit BLDC und Encoder. \cite[S.1]{trinamicmotion_control_gmbh__co_kg_tmc6200_2019}}
	\label{fig:Blockdiagramm_TMC4671_und_TMC6200}
\end{figure}

Es wurde darauf geachtet, dass der Aufbau des Prints dem Testaufbau entspricht. In Abbildung \ref{fig:Blockdiagramm_Motorengruppe} wird ein detailierteres Blockschaltbild gezeigt, welches den Aufbau eher nach Funktionen beschreibt.

\begin{figure}[H]
	\centering
	\includegraphics[width=0.8\textwidth]{graphics/Blockdiagramm_Motorengruppe}
	\caption{Blockschaltbild Motorengruppe nach Funktionen. \cite[S.1]{trinamicmotion_control_gmbh__co_kg_tmc4671_2019}}
	\label{fig:Blockdiagramm_Motorengruppe}
\end{figure}

%\begin{table}[H]
%\center
%\begin{tabular}{|lll|l|}
%\hline
%\textbf{Kennzeichnung} & & \textbf{Bauteil} & \textbf{Funktion} \\
%\hline
%TMC4671 & = & Trinamic TMC4671 & FOC-Treiber \\
%GATE driver & = & Trinamic TMC6200 & Gate-Treiber \\
%POWER stage & = & Trnamic UPS 10A70V & H-Brücke \\
%Current sensing & = & UPS 10A70V ==> TMC6200 & Messung Phasenströme \\
%PMSM/BLDC & = & Sigmatec AKM22h & BLDC \\
%ENCODER & = & CUI devices ATS33 & ABN-Encoder \\
%\hline
%\end{tabular}
%\end{table}
%\newpage