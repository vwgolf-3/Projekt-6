\clearpage
\section{Zielerreichung}
\label{sec:Zielerreichung}





\subsection{Pflichzielerfüllung}
\label{subsubsec:Pflichtzielerfüllung}

\begin{figure}[H]
	\begin{flushleft}
	\small
		\begin{tabular}{|p{3cm}|p{3.25cm}|p{9.85cm}|}%{|c|l|l|}
\hline
\multicolumn{1}{|l|}{\textbf{Nummer}} & \textbf{Pflichtziele}  & \textbf{Anforderungen}                                                                                                                                            \\ \hline

\multicolumn{1}{|c|}{\text{\cellcolor{green}1}} & \cellcolor{green} 

Detailkonzept & \cellcolor{green}Das Detailkonzept besteht aus folgenden Komponenten:
\begin{itemize}
\item Speisungen (48V, 12V, 5V, 3.3V)
\item Motor
\item ABN-Encoder
\item Endschalten
\item Motorentreiber
\item Gate-Treiber
\item Durchflussmessungen
\item Pumpen
\item Display
\item Mikrocontroller
\item Programmierschnittstellen
\item Bluetooth-/Wirelessmodul
\item Beleuchtung
\end{itemize} \\ \hline

\multicolumn{1}{|c|}{\cellcolor{green}2} & \cellcolor{green}Design der Leiterplatte & \cellcolor{green}Soll alle Teile des Detailkonzeptes umfassen. Für das Wifi-/RFID- und Motorentreibermodul wird eine Development-Board verwendet. Zusätzlich zu WiFi- und RFID-Modul wird eine eigen gelayoutete Variante miteinbezogen, welche bei genügend Kapazität implementiert wird anstelle des Moduls.\newline
\\ \hline

\multicolumn{1}{|c|}{\cellcolor{green}3} & \cellcolor{green}Mechanischer Aufbau der Maschine inkl. Achsensystem. & \cellcolor{green}Der mechanische Aufbau beinhaltet folgende Teile:

\begin{itemize}
\item Rahmen
\item Getränkehalterung
\item Flüssigkeitsbeförderung
\item Gehäuse für Elektronik
\item Befestigung für Display
\item Glasbeförderungssystem
\item Überlaufwanne
\item Beleuchtung
\end{itemize} \\ \hline

\multicolumn{1}{|c|}{\cellcolor{green}4} & \cellcolor{green}Regler Parametrisierung des Achsensystems & \cellcolor{green}Die Regelung des Achsensystems wird mit dem FOC-Treiber gewährleistet. Die Regler werden so ausgelegt, dass das Glas während dem Fahren nicht überläuft. Die Bewegungsgeschwindigkeit soll trotzdem genügend schnell sein, dass der Drink in unter einer Minute hergestellt wird.\\ \hline

\multicolumn{1}{|c|}{\cellcolor{green}5} & \cellcolor{green}Bediensoftware & \cellcolor{green}Die Bediensoftware auf dem Mikrocontroller ermöglicht dem Benutzer folgende Eingabe:
\begin{itemize}
\item Getränkeliste mit 5 alkoholischen und 5 nichtalkoholischen Getränken, welche zur Auswahl stehen.
\item Infos zu den Getränken
\item Auswahl der Zubereitungsgrösse von 3 oder 5dl
\item Nachfüllen des per Webservers (oder vorzugsweise per Android-App gemäss Wunschziel) eingestellten Getränkes mittels RFID.
\item Reinigungsmodus
\end{itemize} 
Über einen Webserver (oder Android-App gemäss Wunschziel) kann der User folgende Einstellungen vornehmen:
\begin{itemize}
\item Zuweisung eines RFID-Tags zu einem Benutzer
\item Auswahl des nächsten Getränkes gemäss der Getränkeliste
\end{itemize}
		\end{tabular}
	\end{flushleft}
	\label{table:Pflichtziele}
	
\end{figure}

\begin{figure}[H]
	\begin{flushleft}
	\small
		\begin{tabular}{|p{3cm}|p{3.25cm}|p{9.85cm}|}%{|c|l|l|}
\hline
\multicolumn{1}{|l|}{\textbf{Nummer}} & \textbf{Pflichtziele}  & \textbf{Anforderungen}                                                                                                                                            \\ \hline

\multicolumn{1}{|c|}{\cellcolor{green}6} & \cellcolor{green}Funktionstest und Analyse bezüglich der Skalierbarkeit & \cellcolor{green} In einer ersten wird der Print in Betrieb genommen. Dies bedeutet, dass die einzelnen Systeme mit Sonderprogrammen auf ihre Funktion geprüft werden. Dies beinhaltet die Systeme des Detailkonzeptes.

In einer zweiten Phase wird die Maschine auf ihre Funktion geprüft. Dies soll die Funktionen beinhalten, welche in der Bediensoftware aufgelistet sind. \\ \hline	

\multicolumn{1}{|c|}{\cellcolor{green}7} & \cellcolor{green}Software & \cellcolor{green}Die Software für den Mikrocontroller soll in C geschrieben sein, für das ESP wird vorerst Arduino verwendet. \\ \hline

\multicolumn{1}{|c|}{\cellcolor{green}8} & \cellcolor{green}Getränkezubereitung & \cellcolor{green}Die Abweichung der Flüssigkeitsausgabe darf höchstens 4\% betragen. \\ \hline	
		\end{tabular}
	\end{flushleft}
	\label{table:Pflichtziele2}
	
\end{figure}

\subsection{Wunschzielerfüllung}
\label{subsubsec:Wunschzielerfüllung}

\begin{figure}[H]
	\begin{flushleft}
		\small
		\begin{tabular}{|p{3cm}|p{3.25cm}|p{9.85cm}|}%{|c|l|l|}
\hline
\multicolumn{1}{|l|}{\textbf{Nummer}} & \textbf{Wunschziele}  & \textbf{Anforderungen}                                                                                                                                         \\ \hline
		
\multicolumn{1}{|c|}{\cellcolor{green}1} & \cellcolor{green}Lichtkonzept & \cellcolor{green} Die Maschine bietet einen gewissen Showeffekt. Dazu wird ein LED-Band montiert, welcher die Maschine beleuchtet. Für die Maschine werden RGB-LEDs verwendet, was eine entsprechende Ansteuerung Hard- und Softwareseitig benötigt. \\ \hline

\multicolumn{1}{|c|}{\cellcolor{orange}2} & \cellcolor{orange}Software & \cellcolor{orange}
\begin{itemize}
\item Die Software für das ESP soll in C geschrieben sein. 
\item Es soll vom Nutzer konfigurierbar sein, wo welches Getränk steht.
\item Der Benutzer soll selber Cocktails individuell erstellen können.
\item Individuelle Anpassungen der Mischverhältnisse der gespeicherten Getränke.
\end{itemize} \\ \hline

\multicolumn{1}{|c|}{\cellcolor{green}3} & \cellcolor{green}Android-Applikation & \cellcolor{green}\begin{itemize}
\item Anstelle des Webservers soll eine Android-App erstellt werden, welche über Bluetooth kommuniziert.
\item Individuelle Erstellung von Cocktails in der App, gemäss Flüssigkeitsliste.
\end{itemize} \\ \hline
			
\multicolumn{1}{|c|}{\cellcolor{green}4} & \cellcolor{green}Regler Parametrisierung des Achsensystems & \cellcolor{green}\begin{itemize}
\item Das gewünschte Getränk soll in unter 40s zubereitet werden.
\end{itemize} \\ 
			\hline
			
			\multicolumn{1}{|c|}{\cellcolor{red}5} & \cellcolor{red}Getränkezubereitung& \cellcolor{red}\begin{itemize}
\item Die Abweichung der Flüssigkeitsabweichung darf höchstens 1\% betragen.
\end{itemize} \\ 
			\hline
		\end{tabular}
	\end{flushleft}

	\label{table:Wunschziele}
\end{figure}
\newpage
