\subsubsection{ESP}
\label{subsubsec:Inbetriebnahme_ESP}


Inbetriebnahme Programmierlogik:

1 = RTS

2 = DTR

3 = EN

4 = IO0



Die Inbetriebnahme des ESP32-WROOM-32U geschieht mit der Programmierumgebung Arduino IDE. Um diesen in Betrieb nehmen zu können, waren einige Einstellungen und Downloads nötig, welche dann in der Arduino IDE eingebunden werden können. Es wird getestet, ob das ESP sich im vorgegebenen Netz anmeldet und als Webhost dient, ob bei Klicken auf ein Button im Explorer ein Event auslöst, ob die Kommunikation zwischen \textmu C und Mikrokontroller funktioniert.

Folgende Schritte wurden befolgt:
\begin{enumerate}
\item Benötigte Daten von Github\footnote{https://github.com/espressif/arduino-esp32} herunterladen.\newline
\item Dateien Entpacken und speichern unter:\newline
\textcolor{magenta}{C:\textbackslash Users \textbackslash ''Benutzer''\textbackslash Documents\textbackslash Arduino\textbackslash hardware \textbackslash espressif \textbackslash esp32}\newline
Damit die Arduino IDE die Files findet.\newline
\item Arduino IDE starten und folgenden Reiter öffnen:\newline
\textcolor{blue}{Arduino IDE \textrightarrow Werkzeuge \textrightarrow Board}\newline
ESP32 Dev Module auswählen.\newline
\item Unter demselben Reiter können noch weitere Einstellungen getätigt werden.\newline
\textcolor{blue}{Arduino IDE \textrightarrow Werkzeuge \textrightarrow ...}\newline
\begin{tabular}{lll}
Upload Speed & : & 921600\\
CPU Frequency & : & 240MHz\\
Flash Frequency & & 80MHz\\
Flash Mode & : & QIO\\
Flash Size & : & 4MB (32Mb)\\
Partition Scheme & : & Default 4MB wit spifss\\
Core Debug Level & : & none\\
PSRAM & : & disabled\\
Port & : & \textcolor{red}{COMx}\\
\end{tabular}

Wobei der Port \textcolor{red}{COMx} im Geräte-Manager ermittelt werden muss. Das ESP32 ist jetzt flashbar.\newline

\item Testprogramm herunterladen\footnote{https://randomnerdtutorials.com/esp32-web-server-arduino-ide/}, leicht modifizieren und hochladen.\newline
\textcolor{blue}{Arduino IDE \textrightarrow Verify and Upload Button} \newline
Für die Inbetriebnahme wurde das Testprogramm so modifiziert, dass das ESP32 gleich getestet werden kann wie das Touch-Display. Der Unteschied ist, dass das ESP über einen anderen UART-Port des \textmu C kommuniziert. Die Testsoftware ist bei den anderen Firmwares auf dem USB-Stick zu finden.\newline
\item Die ''Debug Kommunikation'' über den ersten Seriellen Port des ESP32 testen.

\textcolor{blue}{Arduino IDE \textrightarrow Tools \textrightarrow Serial Monitor}\newline
Im Testprogramm wird hier angezeigt, wenn ein neuer Client die IP-Adresse aufruft und wenn im Internetexplorer ein Button gedrückt wird. Erste Versuche nach dem hochladen waren erfolgreich, das ESP hat eine IP-Adresse zugeordnet bekommen.
\end{enumerate}

Über die zweite serielle Schnittstelle findet die Kommunikation zwischen ESP32 und Atmega2560 statt. Sobald über den Webserver eine Aktion ausgelöst wird, sendet das ESP die gleiche Zeichenfolge, welche auch das Display ausgibt. Der Test zeigt, dass das Drücken auf den Button im Internetexplorer die gleiche Aktion auslöst, wie das Drücken auf das Touch-Display. Es werden Bilder neu gesetzt und Texte geändert. Die Implementierung des EPS ist folglich erfolgreich.