\subsubsection{WiFi-Modul}
\label{subsubsec:Inbetriebnahme_ESP}

Die Inbetriebnahme des WiFi-Moduls geschieht mit der Programmierumgebung Arduino IDE. Damit das WiFi-Modul programmiert werden kann, muss es in der Arduino IDE bekannt gemacht werden. Das Programmer-Tool \textit{esptool.py} ist von espressif und muss ebenfalls in die IDE gelinkt werden.

Im Folgenden wird beschrieben wie vorgegangen wird, damit sich das WiFi-Modul programmieren lässt. Es wird ein Code hochgeladen mit dem sich das ESP als Webhost im vorgegebenen Netz anmeldet. Es wurde getestet, ob ein Klicken auf ein Button im Explorer ein Event auslöst und ob die Kommunikation zwischen Mikrocontroller und WiFi-Modul funktioniert.

Folgende Schritte wurden befolgt:
\begin{enumerate}
\item Benötigte ESP32-Bibliotheken vom \textit{espressif}-Github-Account runterladen.\cite{grokhotkov_espressifarduino-esp32_2020} \newline
\textcolor{orange}{Github search: espressif/arduino-esp32} \cite{grokhotkov_espressifarduino-esp32_2020}
\newline
%\item Dateien Entpacken und speichern unter:\newline
%\textcolor{magenta}{C:\textbackslash Users\textbackslash ''Benutzer''\textbackslash Documents\textbackslash Arduino\textbackslash hardware\textbackslash espressif\textbackslash esp32}\newline
%Damit die Arduino IDE die Files findet.\newline

\item Speichern der heruntergeladenen Daten unter:\newline
\textcolor{blue}{.../Dokumente/Arduino/hardware/espressif/esp32/\textit{(Inhalt: cores, docs, libraries, ...)}}\newline
%\item Arduino IDE starten und Einbinden des ESP32-Dev-Kit's unter:\newline
%\textcolor{blue} {Werkzeuge \textrightarrow Board \textrightarrow Boardverwalter \textrightarrow esp32} \newline
Danach sollte das WiFi-Modul in der Arduino-IDE erscheinen (im sketchbook). Dort kann nun das ESP32 Dev Module angewählt werden.\newline
\textcolor{blue} {Werkzeuge \textrightarrow Board}\newline
%\item Arduino IDE starten und folgenden Reiter öffnen:\newline
%\textcolor{blue}{Arduino IDE \textrightarrow Werkzeuge \textrightarrow Board}\newline
%ESP32 Dev Module auswählen.\newline
\item Unter dem selben Reiter können noch weitere Einstellungen getätigt werden.\newline
\textcolor{blue}{Arduino IDE \textrightarrow Werkzeuge \textrightarrow ...}\newline
\begin{tabular}{lll}
Upload Speed & : & 921600\\
CPU Frequency & : & 240MHz\\
Flash Frequency & & 80MHz\\
Flash Mode & : & QIO\\
Flash Size & : & 4MB (32Mb)\\
Partition Scheme & : & Default 4MB wit spifss\\
Core Debug Level & : & none\\
PSRAM & : & disabled\\
Port & : & \textcolor{red}{COMx}\\
\end{tabular}

Wobei der Port \textcolor{red}{COMx} im Geräte-Manager ermittelt werden muss. Das WiFi-Modul ist jetzt computerseitig flashbar.\newpage

\item Testprogramm aus dem Blog-Post von Randomnerdtutorials.com herunterladen, leicht modifizieren und hochladen. \cite{santos_esp32_2018}\newline
\textcolor{blue}{Arduino IDE \textrightarrow Verify and Upload Button} \newline
Für die Inbetriebnahme wurde das Testprogramm so modifiziert, dass das WiFi-Modul gleich getestet werden kann wie das Touch-Display. Dies bedeutet, dass das WiFi-Modul eine Page-ID, eine Button-ID und das Abschlusszeichen 0xFF 0xFF 0xFF sendet. Der Unterschied ist, dass das WiFi-Modul über einen anderen UART-Port des Mikrocontrollers kommuniziert. Die Testsoftware ist bei den anderen Firmwares auf dem USB-Stick zu finden.\newline

\item Die ''Debug Kommunikation'' über den ersten Seriellen Port des WiFi-Moduls testen.

\textcolor{blue}{Arduino IDE \textrightarrow Tools \textrightarrow Serial Monitor}\newline
Im Testprogramm wird hier angezeigt, wenn ein neuer Client die IP-Adresse aufruft und wenn im Internetexplorer ein Button gedrückt wird. Erste Versuche nach dem Hochladen waren erfolgreich. Das WiFi-Modul hat eine IP-Adresse zugeordnet bekommen.
\end{enumerate}

Ergebins: Der Klick auf das Display bewirkt, dass eine Kommunikation zwschen WiFi-Modul und Mikrocontroller stattfindet. Der Test zeigt, dass das Drücken auf den Button im Internetexplorer die gleiche Aktion auslöst, wie das Drücken auf das Touch-Display. Es werden Bilder neu gesetzt und Texte geändert. Die Implementierung des EPS ist folglich erfolgreich. Im Anhang Kapitel \ref{Appendix:ESP32_Arduino_IDE} sind einige Auszüge aus der Arduino IDE aufgelistet.