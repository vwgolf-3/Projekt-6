\newpage
\subsection{Motor}
\label{subsec:Inbetriebnahme_Motor}

Der Motor wird in folgender Reihenfolge in Betrieb genommen:

\begin{enumerate}
\item FOC-Treiber
\item Gate-Treiber
\item H-Brücke
\item ABN-Encoder
\item PI-Regler
\end{enumerate}

Da der FOC-Treiber in einem Openloop-Modus betrieben werden kann, eignet es sich, diesen als erstes in Betrieb zu nehmen. Im Openloop werden Schaltsignale erzeugt, welche den Motor mit einer vorbestimmten Drehzahl laufen lassen, ohne das Verhalten des Motors zu messen, weder den Strom durch die Spulen noch Lage des Rotors. Werden diese Gatesignale korrekt ausgegeben, so können diese Verwendet werden, den Gate-Treiber in Betrieb zu nehmen. Hand in Hand mit dem Gate-Treiber wird die H-Brücke in Betrieb genommen. Sobald die Openloop-Kette (Mikrocontroller, FOC-Treiber, Gate-Treiber, H-Brücke) funktioniert und der angeschlossene Motor gemäss der Vorgegebenen Geschwindigkeit dreht, kann der ABN-Encoder in Betrieb genommen werden und die Closed-Loop-Modi implementiert werden, welche einen Torque-, Velocty- und Positions-Regelkreis beinhalten.
