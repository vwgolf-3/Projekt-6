\subsubsection{48V Speisung}
\label{subsubec:48V Speisung}


Der Motor wird mit einer Spannung von 48V betrieben. Dies ist zugleich auch die höchste verwendete Speisespannung. Um diese Speisung gewährleisten zu können, wird ein fertiges Netzteil von Aliexpress gemäss  \textcolor{red}{\textbf{Fachbericht 5}} eingesetzt. Somit entfällt das Schema für diesen Speisungsteil. Ein Anschauungsbild des eingesetzten Netzteils kann in Abbildung \ref{fig:Netzteil_48V} begutachtet werden \cite{aliexpress_us_nodate}.\\

\begin{figure}[h!]
	\centering
	\includegraphics[width=0.5\textwidth]{graphics/Netzteil_48V.png}
	\caption{Anschauungsbild des 48V Netzteils \cite{aliexpress_us_nodate}}
	\label{fig:Netzteil_48V}
\end{figure} 

Eine Leistungsabschätzung war jedoch unabdingbar. Auch diese wurde im Projekt 5 durchgeführt. Unter Berücksichtigung der Schaltungsteile welche noch im Projekt 6 ergänzt werden, wurde dieses dann ausgewählt und eingekauft. Es hat sich dabei herausgestellt, dass das Netzteil mindestens eine Leistung von 150W bieten muss. Die Leistungsabschätzung kann im \textcolor{red}{\textbf{Fachbericht 5}} eingesehen werden. 

\newpage