\newpage
\subsection{Datenspeicherung}
\label{sec:Inbetriebnahme_Datenspeicherung}

Im Folgenden wird die Datenspeicherung in Betrieb genommen. Dies geschieht mit dem Programm, welches in der Library vorhanden ist. Damit kann auf das Directory Table zugegriffen werden und die darauf vorhandenen Files gelesen und bearbeitet werden. Es können auch neue Files generiert werden.

Die benutzte Library ist eine ältere Version. Die neueren Versionen sind auf der Homepage gezeigt und bieten einen noch breiteren Verwendungszweck. Ein wirklich gelungenes Projekt. \cite{dharmani_sdsdhc_2009}