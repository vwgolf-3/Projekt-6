\subsubsection{BLDC}
\label{subsubsec:BLDC}

Ein Brushless DC-Motor\footnote{Bürstenloser Gleichstrommotor} (BLDC) zeichnet sich dadurch aus, dass der Rotor mit Permanentmagneten bestückt ist. Somit ähnelt er vom Aufbau her einer Synchronmaschine mit permanent erregten Rotorwicklungen. Zur Ansteuerung hat der BLDC drei Phasenleitungen, die auf die Spulen führen. Die Spulen sind innerhalb des BLDC in Stern geschaltet. Der Motor hat drei Polpaare, womit die magnetische Winkelgeschwindigkeit drei Mal schneller ist als die mechanische.
%Ein Nachteil von BLDC-Motoren ist, dass bei Drehgeschwindigkeiten über Nenndrehzahl nur mit einer geeigneten Regelung der Statorwicklungen in Feldschwächung gefahren werden kann, anstelle dass der Erregerstrom verringert wird. Ansonsten zeichnet sich ein BLDC durch ein besseres Leistungs-/Gewichtsverhältnis aus als herkömmliche Motoren.

\paragraph{Schema}\mbox{}

Zum BLDC gibt es kein Schema. Die Anschlusspins für die Motorleitungen sind in Abbildung \ref{fig:Schema_H_Bruecke_und_BLDC} zu sehen.

\paragraph{Funktionsbeschrieb der Schaltung}\mbox{}

Im Folgenden werden die Eigenschaften des verwendeten BLDCs aufgelistet. So wurde beispielsweise die Versorgungsspannung nach der Nennspannung $V_M$ ausgewählt, das Limit des Geschwindigkeitsregelkreis auf 1500min$^{-1}$ gesetzt, der Strom auf 5A begrenzt \cite[S.36]{akm_servomotoren_2020}

\begin{table}
\begin{tabularx}{\textwidth}{|lllllX|}
\hline
\textbf{Beschreibung} & \textbf{Var} & & \textbf{Wert} & \textbf{Einheit} & \textbf{Benötigt für} \\
\hline
Nennnetzspannung & $V_M$ & = & 48 & [V] & Versorgungsspannung\\
Stillstanddrehmoment & $M_0$ & = & 0.88 & [Nm] & -\\
Nenndrehmoment & $M_n$ & = & 0.85 & [Nm] & -\\
Spitzendrehmoment & $M_{max}$ & = & 2.8 & [Nm] & -\\
Nenndrehzahl & $n_n$ & = & 1500 & [min$^{-1}$] & Limit: Geschwindigkeit (PI-Regler)\\
Nennleistung & $P_n$ & = & 0.13 & [kW] & Power Management\\
Stillstandstrom & $I_0$ & = & 5.41 & [A] & -\\
Nennstrom & $I_N$ & = & 5.21 & [A] & Limit: Strom (PI-Regler)\\
Spitzenstrom & $I_{max}$ & = & 21.6 & [A] & -\\
Drehmomentkonstante & $k_T$ & = & 0.1632 & [Nm/A] & Reglerauslegung\\
Rotorträgheitsmoment & $J$ & = & 0.16 & [kg$\cdot$cm$^2$] & Reglerauslegung\\
Gewicht & $m$ & = & 1.1 & [kg] & Mechanik\\
\hline
\end{tabularx}
\caption{Charakteristische Eigenschaften des BLDC}
\end{table}
